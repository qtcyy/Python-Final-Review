\documentclass[11pt,a4paper]{article}

\usepackage{ctex}
\usepackage[utf8]{inputenc}
\usepackage{graphicx}
\usepackage{subcaption}
\usepackage{multirow}
\usepackage{enumitem}
\usepackage{geometry}
\usepackage{hyperref}
\hypersetup{
  colorlinks=true,  % 启用彩色链接
  linkcolor=blue,   % 内部链接颜色(目录、交叉引用等)
  urlcolor=blue,    % URL链接颜色
  citecolor=blue,   % 引用链接颜色
  bookmarksdepth=3, % PDF书签深度
  pdfstartview=FitH % PDF打开时的视图方式
}

\geometry{margin=2.5cm}

\usepackage{listings}
\usepackage[dvipsnames]{xcolor}
\usepackage[framemethod=TikZ]{mdframed}
\usepackage{tikz}  % 添加TikZ包用于绘图
\usetikzlibrary{shapes,positioning}
% 在导言区进行样式设置
\lstset{
  language=Python, % 设置语言
  basicstyle=\ttfamily\small, % 设置字体族和大小
  breaklines=true, % 自动换行
  keywordstyle=\bfseries\color{Purple}, % C语言关键字:紫色粗体
  morekeywords={uint32_t,uint8_t,float,char,static,const}, % C语言扩展关键字
  emph=[1]{delay_init,key_init,lcd_init,infrared_init,rgb_init,buzzer_init,led_init,usart_init,htu21d_init,fan_pwm_init}, % 初始化函数
  emphstyle=[1]\bfseries\color{Red}, % 初始化函数:红色粗体
  emph=[2]{htu21d_t,htu21d_h,get_infrared_status,fan_pwm_control,rgb_ctrl,led_control,buzzer_tweet}, % 硬件控制函数
  emphstyle=[2]\bfseries\color{Orange}, % 硬件控制函数:橙色粗体
  emph=[3]{update_display,get_centered_x,check_uart_command,send_system_status,auto_control,check_human_presence}, % 逻辑函数
  emphstyle=[3]\bfseries\color{Blue}, % 逻辑函数:蓝色粗体
  emph=[4]{sprintf,printf,strncmp,strlen,delay_ms,delay_count}, % 标准库函数
  emphstyle=[4]\color{Magenta}, % 标准库函数:洋红色
  commentstyle=\itshape\color{ForestGreen}, % 注释:森林绿斜体
  stringstyle=\color{Maroon}, % 字符串:栗色
  identifierstyle=\color{Black}, % 标识符:黑色
  numberstyle=\tiny\color{Gray}, % 数字:灰色小字
  columns=flexible,
  numbers=none, % 不显示行号
  % numbersep=1.5em, % 设置行号的具体位置
  % numberstyle=\footnotesize\color{Gray}, % 行号样式
  frame=single, % 边框
  framesep=1em, % 设置代码与边框的距离
  rulecolor=\color{lightgray}, % 边框颜色
  backgroundcolor=\color{Gray!5}, % 背景色:淡灰色
  tabsize=4, % Tab键宽度
  showstringspaces=false % 不显示字符串中的空格标记
}

\begin{document}

\begin{minipage}[t]{0.7\textwidth}
  \fontsize{20pt}{24pt}\selectfont Python 期末复习模拟卷笔记
\end{minipage}
\begin{minipage}[p]{0.25\textwidth}
  \hfill
  \includegraphics[width=3cm]{./sticker.pdf}
\end{minipage}

\tableofcontents

\newpage

\vspace{1cm}

\raggedright

\section{单选题}

\subsection{第1题}
下面代码输出的结果是()
\begin{lstlisting}
def func(num):
    num += 1
a = 10
func(a)
print(a)
\end{lstlisting}

\begin{enumerate}[label=\Alph*.]
  \item 10
  \item 11
  \item int
  \item 程序执行错误
\end{enumerate}

\textbf{\color{red}【笔记】}

正确答案:A

Python中函数是传值调用,不影响原本的值。

\subsection{第2题}
切片操作\texttt{list(range(6))[::-1]}执行结果为()

\begin{enumerate}[label=\Alph*.]
  \item \verb|[|0, 2, 4, 6\verb|]|
  \item \verb|[|6, 5, 4, 3, 2, 1\verb|]|
  \item \verb|[|0, -1\verb|]|
  \item \verb|[|5, 4, 3, 2, 1, 0\verb|]|
\end{enumerate}

\begin{mdframed}[linewidth=1pt, linecolor=black]

  \textbf{\color{red}【笔记】}

  正确答案:D

  \texttt{list()}将对象转换成列表类型,\texttt{[::-1]}将列表反转。

\end{mdframed}

\subsection{第3题}
下列选项中可以获取Python整数类型帮助的是()

\begin{enumerate}[label=\Alph*.]
  \item help(float)
  \item dir(float)
  \item help(int)
  \item dir(str)
\end{enumerate}

\subsection{第4题}
以下选项中符合Python语言变量命名规则的是()

\begin{enumerate}[label=\Alph*.]
  \item it's
  \item 3C
  \item pass
  \item \_AI
\end{enumerate}

\begin{mdframed}[linewidth=1pt, linecolor=black]

  \textbf{\color{red}【笔记】}

  正确答案:D

  \begin{itemize}
    \item A选项中出现了\texttt{'}符号,在Python中的意思是字符串;
    \item B选项数字开头是不允许的;
    \item C选项是Python关键字;
    \item D选项符合Python变量名要求。
  \end{itemize}

\end{mdframed}

\subsection{第5题}
关于Python序列类型的通用操作符和函数,以下选项中描述错误的是()

\begin{enumerate}[label=\Alph*.]
  \item 如果x是s的元素,x in s 返回True
  \item 如果s是一个序列,s = [1,"kate",True],s[3] 返回True
  \item 如果s是一个序列,s = [1,"kate",True],s[-1] 返回True
  \item 如果x不是s的元素,x not in s 返回True
\end{enumerate}

\begin{mdframed}[linewidth=1pt, linecolor=black]

  \textbf{\color{red}【笔记】}

  正确答案:B

  列表s的长度为3,可以访问的索引为0,1,2,如果访问s[3]会报错。

\end{mdframed}

\subsection{第6题}
下列选项中不是Python保留字的是()

\begin{enumerate}[label=\Alph*.]
  \item False
  \item True
  \item do
  \item class
\end{enumerate}

\begin{mdframed}[linewidth=1pt, linecolor=black]

  \textbf{\color{red}【笔记】}

  正确答案:C

  do是C语言关键字,Python中无此关键字。

\end{mdframed}

\subsection{第7题}
Python 3.x语言中,以下表达式输出结果为66的选项是()

\begin{enumerate}[label=\Alph*.]
  \item print("6+6")
  \item print(6+6)
  \item print(eval("6+6"))
  \item print(eval("6" + "6"))
\end{enumerate}

\begin{mdframed}[linewidth=1pt, linecolor=black]

  \textbf{\color{red}【笔记】}

  正确答案:D

  \begin{itemize}
    \item A选项打印的是字符串"6+6";
    \item B选项打印的是6+6的值12;
    \item C选项使用eval将两个字符串'6'转换成数字6后相加,得到12并输出;
    \item D选项使用eval将字符串'66'转换成数字66并输出。
  \end{itemize}

\end{mdframed}

\subsection{第8题}
关于eval函数,以下选项中描述错误的是()

\begin{enumerate}[label=\Alph*.]
  \item eval函数的作用是将输入的字符串转为Python语句,并执行该语句
  \item 如果用户希望输入一个数字,并用程序对这个数字进行计算,可以采用eval(input())组合
  \item 执行eval("Hello")和执行eval("'Hello'")得到相同的结果
  \item 执行eval('123')输出123
\end{enumerate}

\begin{mdframed}[linewidth=1pt, linecolor=black]

  \textbf{\color{red}【笔记】}

  正确答案:C

  \textbf{eval函数详解:}

  \textbf{1. 基本功能:}
  \begin{itemize}
    \item eval()函数用于执行字符串表达式,并返回表达式的结果
    \item 语法:eval(expression, globals=None, locals=None)
    \item expression:要执行的字符串表达式
  \end{itemize}

  \textbf{2. 使用示例:}
    \begin{lstlisting}
# 数学表达式
eval("2 + 3")          # 返回 5
eval("2 * 3 + 1")      # 返回 7
eval("abs(-5)")        # 返回 5

# 字符串表达式
eval("'Hello'")        # 返回 'Hello'
eval("'3' + '4'")      # 返回 '34'

# 变量表达式
x = 10
eval("x + 5")          # 返回 15

# 列表和字典
eval("[1, 2, 3]")      # 返回 [1, 2, 3]
eval("{'a': 1, 'b': 2}")  # 返回 {'a': 1, 'b': 2}
    \end{lstlisting}

  \textbf{3. 错误示例分析:}
  \begin{itemize}
    \item eval("Hello"):会报错,因为Hello被当作变量名,但未定义
    \item eval("'Hello'"):正确,返回字符串'Hello'
    \item eval("123"):返回整数123
    \item eval("'123'"):返回字符串'123'
  \end{itemize}

  \textbf{4. 常见用法:}
    \begin{lstlisting}
# 用户输入处理
user_input = input("请输入表达式: ")
result = eval(user_input)  # 注意:有安全风险

# 配置文件解析
config_str = "{'debug': True, 'port': 8080}"
config = eval(config_str)
    \end{lstlisting}
\end{mdframed}

\subsection{第9题}
下面代码的输出结果是()
\begin{lstlisting}
d = {"大海":"蓝色", "天空":"灰色", "大地":"黑色"}
print(d["大地"], d.get("大地","黄色"), d.setdefault('草地','绿色'))
\end{lstlisting}

\begin{enumerate}[label=\Alph*.]
  \item 黑色 黑色 None
  \item 黑色 黄色 绿色
  \item 黑色 黄色 None
  \item 黑色 黑色 绿色
\end{enumerate}

\begin{mdframed}[linewidth=1pt, linecolor=black]

  \textbf{\color{red}【笔记】}

  正确答案:D

  \begin{itemize}
    \item 第一个d["大地"]在字典d中存在键,对应为"蓝色";
    \item 第二个d.get("大地","黄色"),其含义为获取键“大地”,如果不存在返回值“黄色”
    \item 第三个d.setdefault('草地','绿色'),key不存在,将key设置为default值,并返回default这个值。
  \end{itemize}

\end{mdframed}

\subsection{第10题}
以下哪个是Python中用于科学计算与可视化的第三方库()

\begin{enumerate}[label=\Alph*.]
  \item jieba
  \item scipy
  \item request
  \item random
\end{enumerate}

\subsection{第11题}
以下选项中,不是Python对文件的打开模式的是()

\begin{enumerate}[label=\Alph*.]
  \item 'r'
  \item 'c'
  \item 'w'
  \item '+'
\end{enumerate}

\begin{mdframed}[linewidth=1pt, linecolor=black]

  \textbf{\color{red}【笔记】}

  正确答案:B

  Python文件打开模式说明:
  \begin{itemize}
    \item 'r':只读模式(默认),文件必须存在
    \item 'w':写入模式,会覆盖原文件内容,如果文件不存在则创建
    \item '+':可读写模式,必须与其他模式组合使用,如'r+'、'w+'
    \item 'c':不是Python的文件打开模式,此选项为错误选项
  \end{itemize}

  其他常见的文件打开模式:
  \begin{itemize}
    \item 'a':追加模式,在文件末尾写入,如果文件不存在则创建
    \item 'x':独占创建模式,文件必须不存在
    \item 'b':二进制模式,与其他模式组合使用,如'rb'、'wb'
    \item 't':文本模式(默认),与其他模式组合使用,如'rt'、'wt'
  \end{itemize}

\end{mdframed}

\subsection{第12题}
关于下面代码中的变量x,以下选项中描述正确的是()
\begin{lstlisting}
fo = open(fname, "r")
for x in fo:
    print(x)
fo.close()
\end{lstlisting}

\begin{enumerate}[label=\Alph*.]
  \item 变量x表示文件中的一组字符
  \item 变量x表示文件中的一行字符
  \item 变量x表示文件中的一个字符
  \item 变量x表示文件中的全体字符
\end{enumerate}

\begin{mdframed}[linewidth=1pt, linecolor=black]

  \textbf{\color{red}【笔记】}

  正确答案:B

  代码分析:
  \begin{itemize}
    \item open(fname, "r")以只读模式打开文件,返回文件对象fo
    \item 在Python中,当使用for循环遍历文件对象时,每次迭代获得的是文件的一行内容
    \item 变量x在每次循环中存储文件的一行字符(包括行尾的换行符)
    \item print(x)会输出文件的每一行内容
    \item fo.close()关闭文件
  \end{itemize}

  知识点:
  \begin{itemize}
    \item 文件对象是可迭代的,for循环遍历文件对象时逐行读取
    \item 每行内容包含该行的所有字符以及行尾的换行符
    \item 这种方式适合处理大文件,因为每次只读取一行到内存中
  \end{itemize}

\end{mdframed}

\subsection{第13题}
以上代码输出结果为()
\begin{lstlisting}
for i in "Python":
    print(i, end=", ")
\end{lstlisting}

\begin{enumerate}[label=\Alph*.]
  \item P*y*t*h*o*n*
  \item P,y,t,h,o,n,
  \item P y t h o n
  \item Python
\end{enumerate}

\begin{mdframed}[linewidth=1pt, linecolor=black]

  \textbf{\color{red}【笔记】}

  正确答案:B

  代码分析:
  \begin{itemize}
    \item for循环遍历字符串"Python",每次取出一个字符赋值给变量i
    \item print(i, end=", ")中的end参数指定输出结束时不换行,而是输出", "
    \item 因此每个字符后面都会跟着逗号和空格
    \item 最终输出结果为:P, y, t, h, o, n,
  \end{itemize}

  知识点:print函数的end参数用于指定输出结束时的字符,默认为换行符\texttt{\\n}。

\end{mdframed}

\subsection{第14题}
执行Python语句\texttt{nums=set([1,2,2,3,3,3,4])}和\texttt{print(len(nums))}的结果是()

\begin{enumerate}[label=\Alph*.]
  \item 1
  \item 2
  \item 4
  \item 7
\end{enumerate}

\begin{mdframed}[linewidth=1pt, linecolor=black]

  \textbf{\color{red}【笔记】}

  正确答案:C

  代码分析:
  \begin{itemize}
    \item 列表[1,2,2,3,3,3,4]包含7个元素,其中有重复元素
    \item set()函数将列表转换为集合,集合具有元素唯一性,会自动去除重复元素
    \item 转换后的集合nums包含元素\{1,2,3,4\}
    \item len(nums)返回集合中元素的个数,即4
  \end{itemize}

  知识点:
  \begin{itemize}
    \item set是Python的内置数据类型,表示无序且不重复的元素集合
    \item set会自动去除重复元素,保持元素的唯一性
    \item len()函数返回序列或集合中元素的个数
  \end{itemize}

\end{mdframed}

\subsection{第15题}
如下代码的输出为()
\begin{lstlisting}
import re
s = 'a bc abc abbb abbbbbca'
re.findall('ab*', s)
\end{lstlisting}

\begin{enumerate}[label=\Alph*.]
  \item \verb|[|'ab', 'ab', 'ab'\verb|]|
  \item \verb|[|'ab', 'abbb', 'abbbb'\verb|]|
  \item \verb|[|'a', 'ab', 'abbb', 'abbbb', 'a'\verb|]|
  \item \verb|[|'a', 'ab', 'abbb', 'abbbb'\verb|]|
\end{enumerate}

\begin{mdframed}[linewidth=1pt, linecolor=black]

  \textbf{\color{red}【笔记】}

  正确答案:C

  \textbf{正则表达式详解:}

  \textbf{1. 基本概念:}
  \begin{itemize}
    \item 正则表达式(Regular Expression)是用于匹配字符串的强大工具
    \item 'ab*'表示匹配字母'a'后跟0个或多个字母'b'
    \item re.findall()返回字符串中所有匹配模式的子字符串列表
  \end{itemize}

  \textbf{2. 量词详解:}
  \begin{itemize}
    \item *:匹配前面字符0次或多次(贪婪匹配)
    \item +:匹配前面字符1次或多次
    \item ?:匹配前面字符0次或1次
    \item \{n\}:恰好匹配n次
    \item \{n,\}:至少匹配n次
    \item \{n,m\}:匹配n到m次
  \end{itemize}

  \textbf{3. 匹配过程分析:}
  \begin{lstlisting}
字符串:'a bc abc abbb abbbbbca'
模式:'ab*'

匹配过程:
位置0: 'a' -> 匹配'a'(0个b)✓
位置2: 'b' -> 不匹配(没有前导'a')
位置3: 'c' -> 不匹配
位置5: 'a' -> 开始匹配
位置6: 'b' -> 匹配'ab'(1个b)✓
位置7: 'c' -> 结束匹配
位置9: 'a' -> 开始匹配
位置10-12: 'bbb' -> 匹配'abbb'(3个b)✓
位置14: 'a' -> 开始匹配
位置15-18: 'bbbb' -> 匹配'abbbb'(4个b)✓
位置19: 'b' -> 继续匹配但遇到'c'
位置20: 'c' -> 结束匹配
位置21: 'a' -> 匹配'a'(0个b)✓

结果:['a', 'ab', 'abbb', 'abbbb', 'a']
  \end{lstlisting}

  \textbf{4. 正则表达式语法详解:}

  \textbf{4.1 基本字符类:}
  \begin{lstlisting}
.     # 匹配任意字符(除换行符)
\d    # 匹配数字[0-9]
\D    # 匹配非数字[^0-9]
\w    # 匹配字母、数字、下划线[a-zA-Z0-9_]
\W    # 匹配非字母数字下划线[^a-zA-Z0-9_]
\s    # 匹配空白字符(空格、制表符、换行符等)
\S    # 匹配非空白字符
  \end{lstlisting}

  \textbf{4.2 自定义字符类:}
  \begin{lstlisting}
[abc]     # 匹配a、b或c中的任意一个
[a-z]     # 匹配任意小写字母
[A-Z]     # 匹配任意大写字母
[0-9]     # 匹配任意数字(等同于\d)
[a-zA-Z]  # 匹配任意字母
[^abc]    # 匹配除a、b、c外的任意字符(取反)
[a-z0-9]  # 匹配小写字母或数字
  \end{lstlisting}

  \textbf{4.3 量词(重复次数):}
  \begin{lstlisting}
*         # 匹配前面的字符0次或多次
+         # 匹配前面的字符1次或多次
?         # 匹配前面的字符0次或1次
{n}       # 匹配前面的字符恰好n次
{n,}      # 匹配前面的字符至少n次
{n,m}     # 匹配前面的字符n到m次
*?        # 非贪婪匹配0次或多次
+?        # 非贪婪匹配1次或多次
??        # 非贪婪匹配0次或1次
  \end{lstlisting}

  \textbf{4.4 锚点(位置匹配):}
  \begin{lstlisting}
^         # 匹配字符串开头
$         # 匹配字符串结尾
\b        # 匹配单词边界
\B        # 匹配非单词边界
\A        # 匹配字符串绝对开头
\Z        # 匹配字符串绝对结尾
  \end{lstlisting}

  \textbf{4.5 分组和选择:}
  \begin{lstlisting}
()        # 分组,创建捕获组
(?:...)   # 非捕获组
|         # 或者(选择)
\1, \2    # 反向引用第1、2个捕获组
(?P<name>...)  # 命名捕获组
(?P=name)      # 引用命名捕获组
  \end{lstlisting}

  \textbf{4.6 特殊字符转义:}
  \begin{lstlisting}
\.        # 匹配字面意思的点号
\*        # 匹配字面意思的星号
\+        # 匹配字面意思的加号
\?        # 匹配字面意思的问号
\[        # 匹配字面意思的左方括号
\]        # 匹配字面意思的右方括号
\\        # 匹配字面意思的反斜杠
\^        # 匹配字面意思的插入符
\$        # 匹配字面意思的美元符号
  \end{lstlisting}

  \textbf{4.7 实用示例:}
  \begin{lstlisting}
# 匹配邮箱
r'\w+@\w+\.\w+'

# 匹配手机号(简单版)
r'1[3-9]\d{9}'

# 匹配IP地址
r'\d{1,3}\.\d{1,3}\.\d{1,3}\.\d{1,3}'

# 匹配HTML标签
r'<[^>]+>'

# 匹配中文字符
r'[\u4e00-\u9fa5]+'

# 匹配URL
r'https?://[^\s]+'
  \end{lstlisting}

  \textbf{5. re模块常用函数:}
  \begin{lstlisting}
import re

text = 'Python 3.8 is great!'

# 查找所有匹配
re.findall(r'\d+', text)        # ['3', '8']

# 搜索第一个匹配
match = re.search(r'\d+', text)  # 返回Match对象或None

# 从字符串开头匹配
match = re.match(r'Python', text)  # 匹配成功

# 替换
result = re.sub(r'\d+', 'X', text)  # 'Python X.X is great!'

# 分割
parts = re.split(r'\s+', text)   # ['Python', '3.8', 'is', 'great!']
  \end{lstlisting}

  \textbf{6. 贪婪匹配vs非贪婪匹配:}
  \begin{lstlisting}
text = 'abbbbbb'

# 贪婪匹配(默认)
re.findall('ab*', text)      # ['abbbbbb'] - 匹配尽可能多的b

# 非贪婪匹配
re.findall('ab*?', text)     # ['a'] - 匹配尽可能少的b
  \end{lstlisting}

  \textbf{知识点总结:}
  \begin{itemize}
    \item 理解量词的含义和用法
    \item 掌握正则表达式的匹配原理
    \item 熟悉re模块的常用函数
    \item 区分贪婪匹配和非贪婪匹配
  \end{itemize}

\end{mdframed}

\subsection{第16题}
以下不能创建一个元组的语句是()

\begin{enumerate}[label=\Alph*.]
  \item tup1 = ()
  \item tup2 = 1,
  \item tup3 = (1)
  \item dict4 = tuple("123")
\end{enumerate}

\begin{mdframed}[linewidth=1pt, linecolor=black]

  \textbf{\color{red}【笔记】}

  正确答案:C

  各选项分析:
  \begin{itemize}
    \item A选项:tup1 = () 创建一个空元组,正确
    \item B选项:tup2 = 1, 创建包含一个元素的元组,逗号是关键,正确
    \item C选项:tup3 = (1) 仅仅是给数字1加括号,结果是整数1,不是元组
    \item D选项:tuple("123") 将字符串转换为元组('1', '2', '3'),正确
  \end{itemize}

  知识点:
  \begin{itemize}
    \item 创建元组时,逗号是关键标识符,不是括号
    \item 单个元素的元组必须在元素后加逗号:(1,) 或 1,
    \item 空元组可以用 () 或 tuple() 创建
    \item tuple()函数可以将其他可迭代对象转换为元组
  \end{itemize}

\end{mdframed}

\subsection{第17题}
在PyCharm中运行整个程序的默认快捷键是()

\begin{enumerate}[label=\Alph*.]
  \item Shift+F10
  \item Ctrl+F10
  \item Shift+Ctrl+F10
  \item Shift+Ctrl+Enter
\end{enumerate}

\begin{mdframed}[linewidth=1pt, linecolor=black]

  \textbf{\color{red}【笔记】}

  正确答案:A

  PyCharm常用快捷键:
  \begin{itemize}
    \item Shift+F10:运行整个程序(Run)
    \item Ctrl+F10:运行当前文件
    \item Shift+Ctrl+F10:运行当前光标所在的配置
    \item Shift+Ctrl+Enter:完成当前语句
  \end{itemize}

  其他常用快捷键:
  \begin{itemize}
    \item Shift+F9:调试程序(Debug)
    \item Ctrl+Shift+F9:调试当前文件
    \item F8:单步执行(调试时)
    \item F9:继续执行(调试时)
  \end{itemize}

\end{mdframed}

\subsection{第18题}
下面的语句中()用来把路径path设置为默认路径

\begin{enumerate}[label=\Alph*.]
  \item os.chdir(path)
  \item os.mkdir(path)
  \item os.isdir(path)
  \item os.listdir(path)
\end{enumerate}

\begin{mdframed}[linewidth=1pt, linecolor=black]

  \textbf{\color{red}【笔记】}

  正确答案:A

  os模块常用函数说明:
  \begin{itemize}
    \item os.chdir(path):改变当前工作目录到指定路径,相当于cd命令
    \item os.mkdir(path):创建一个新目录
    \item os.isdir(path):判断路径是否为目录,返回True或False
    \item os.listdir(path):返回指定目录下的文件和目录列表
  \end{itemize}

  其他相关函数:
  \begin{itemize}
    \item os.getcwd():获取当前工作目录
    \item os.makedirs(path):递归创建目录
    \item os.path.exists(path):判断路径是否存在
    \item os.path.join():连接路径
  \end{itemize}

\end{mdframed}

\subsection{第19题}
在Python中,下列说法正确的是()

\begin{enumerate}[label=\Alph*.]
  \item 0xad是合法的十六进制数字表示形式
  \item 3+4j不是合法的Python表达式
  \item 可以使用if作为变量名
  \item 0o12f是合法的八进制数字
\end{enumerate}

\begin{mdframed}[linewidth=1pt, linecolor=black]

  \textbf{\color{red}【笔记】}

  正确答案:A

  各选项分析:
  \begin{itemize}
    \item A选项:0xad是合法的十六进制数字,0x前缀表示十六进制,ad是有效的hex数字
    \item B选项:3+4j是合法的复数表达式,j表示虚数单位
    \item C选项:if是Python保留字(关键字),不能用作变量名
    \item D选项:0o12f不合法,八进制数字只能包含0-7,f不是有效的八进制数字
  \end{itemize}

  Python数字表示形式:
  \begin{itemize}
    \item 十进制:直接写数字,如123
    \item 二进制:0b前缀,如0b1010
    \item 八进制:0o前缀,如0o777(只能包含0-7)
    \item 十六进制:0x前缀,如0xff(可包含0-9和a-f)
    \item 复数:实部+虚部j,如3+4j
  \end{itemize}

\end{mdframed}

\section{判断题}

\subsection{第1题}
定义函数时,即便该函数不需要接收任何参数,也必须保留一对空的圆括号来表示这是一个函数。(\quad)

\begin{mdframed}[linewidth=1pt, linecolor=black]
  \textbf{\color{red}【笔记】}

  正确答案:正确 (T)

  解释:
  \begin{itemize}
    \item Python函数定义语法:\texttt{def function\_name():}
    \item 即使函数不需要参数,圆括号 \texttt{()} 也是必须的
    \item 圆括号是函数定义的语法要求,用于区分函数和变量
    \item 示例:\texttt{def hello(): print("Hello")}
  \end{itemize}

\end{mdframed}

\subsection{第2题}
已知\texttt{ls=[2, 4, 6]},那么执行语句\texttt{ls.append(8)}之后,ls的内存地址不变。(\quad)

\begin{mdframed}[linewidth=1pt, linecolor=black]
  \textbf{\color{red}【笔记】}

  正确答案:正确 (T)

  解释:
  \begin{itemize}
    \item \texttt{append()}方法是原地修改操作(in-place operation)
    \item 列表对象本身不会被重新创建,只是在原有内存空间中添加新元素
    \item 可以用\texttt{id()}函数验证:\texttt{id(ls)}在append前后相同
    \item 类似的原地修改方法还有:\texttt{extend()}, \texttt{insert()}, \texttt{remove()}等
  \end{itemize}

  对比:重新赋值操作如\texttt{ls = ls + [8]}会创建新的列表对象,内存地址会改变。

\end{mdframed}

\subsection{第3题}
continue语句在一旦在循环结构里被执行,将使得当前循环的整个循环提前结束。(\quad)

\begin{mdframed}[linewidth=1pt, linecolor=black]
  \textbf{\color{red}【笔记】}

  正确答案:错误 (F)

  解释:
  \begin{itemize}
    \item \texttt{continue}语句只是跳过当前循环迭代,继续下一次迭代
    \item \texttt{continue}不会终止整个循环,循环仍会继续执行
    \item \texttt{break}语句才会提前终止整个循环
  \end{itemize}

  示例对比:
  \begin{itemize}
    \item 使用\texttt{continue}:跳过满足条件的迭代,继续其他迭代
    \item 使用\texttt{break}:一旦满足条件就立即退出整个循环
  \end{itemize}

\end{mdframed}

\subsection{第4题}
\texttt{a,b=b,a}可以实现a和b值的交换。(\quad)

\begin{mdframed}[linewidth=1pt, linecolor=black]
  \textbf{\color{red}【笔记】}

  正确答案:正确 (T)

  解释:
  \begin{itemize}
    \item Python支持多重赋值(multiple assignment)
    \item 右侧\texttt{b,a}先被打包成元组\texttt{(b,a)}
    \item 然后元组被解包赋值给左侧的\texttt{a,b}
    \item 这种方式简洁高效,无需临时变量
  \end{itemize}

  传统交换方式对比:
  \begin{itemize}
    \item 传统方式:\texttt{temp=a; a=b; b=temp}(需要临时变量)
    \item Python方式:\texttt{a,b=b,a}(一行代码完成)
  \end{itemize}

\end{mdframed}

\subsection{第5题}
函数是封装了一些独立的功能,可以直接调用,Python内置了许多函数,同时可以自建函数来使用。(\quad)

\begin{mdframed}[linewidth=1pt, linecolor=black]
  \textbf{\color{red}【笔记】}

  正确答案:正确 (T)

  解释:
  \begin{itemize}
    \item 函数是代码重用和模块化的基础
    \item 函数封装特定功能,提高代码可读性和可维护性
    \item Python内置函数如:\texttt{print()}, \texttt{len()}, \texttt{max()}, \texttt{min()}等
    \item 用户可以使用\texttt{def}关键字自定义函数
  \end{itemize}

  函数的优点:
  \begin{itemize}
    \item 代码重用:避免重复编写相同代码
    \item 模块化:将复杂问题分解为小问题
    \item 易于测试和调试
  \end{itemize}

\end{mdframed}

\subsection{第6题}
使用内置函数\texttt{open()}且以'w'模式打开的文件,文件指针默认指向文件尾。(\quad)

\begin{mdframed}[linewidth=1pt, linecolor=black]
  \textbf{\color{red}【笔记】}

  正确答案:错误 (F)

  解释:
  \begin{itemize}
    \item 'w'模式(写入模式)文件指针指向文件开头
    \item 'w'模式会清空原文件内容,从头开始写入
    \item 'a'模式(追加模式)文件指针才指向文件尾
  \end{itemize}

  文件打开模式对比:
  \begin{itemize}
    \item 'r':只读模式,指针在文件开头
    \item 'w':写入模式,清空文件,指针在文件开头
    \item 'a':追加模式,指针在文件末尾
    \item 'r+':读写模式,指针在文件开头
  \end{itemize}

\end{mdframed}

\subsection{第7题}
利用\texttt{print()}格式化输出,\verb|{2:f}|能够控制浮点数的小数点后保留两位。(\quad)

\begin{mdframed}[linewidth=1pt, linecolor=black]
  \textbf{\color{red}【笔记】}

  正确答案:错误 (F)

  解释:
  \begin{itemize}
    \item \verb|{2:f}|不能控制小数点后保留两位
    \item 2表示参数索引(第3个参数,从0开始计数)
    \item f表示浮点数格式,但没有指定精度
    \item 要保留两位小数需要使用\verb|{2:.2f}|
  \end{itemize}

  正确的格式化语法:
  \begin{itemize}
    \item \verb|{2:.2f}|:第3个参数保留2位小数
    \item \verb|{:.2f}|:当前位置参数保留2位小数
    \item \verb|{2:f}|:第3个参数浮点数格式,但精度由系统默认决定
  \end{itemize}

  \label{string:format}
  \textbf{Python print()格式化输出详解:}

  \textbf{1. 基本格式化语法}
  \begin{itemize}
    \item 基本形式:\verb|print("格式字符串".format(参数))|
    \item f-string形式:\verb|print(f"格式字符串{变量}")| (Python 3.6+)
    \item \%格式化:\verb|print("格式字符串" % 参数)| (较老的方式)
    \end{itemize}

    \textbf{2. 格式说明符详解}
    \begin{itemize}
      \item \verb|{索引:格式说明符}|
      \item 索引:0, 1, 2... 或省略(按顺序)
      \item 格式说明符:[填充字符][对齐][宽度][.精度][类型]
    \end{itemize}

    \textbf{3. 数字格式化类型}
    \begin{itemize}
      \item \texttt{d}:十进制整数
      \item \texttt{f}:浮点数(默认6位小数)
      \item \texttt{.nf}:浮点数保留n位小数
      \item \texttt{e}:科学计数法
      \item \texttt{g}:自动选择f或e格式
      \item \texttt{o}:八进制,\texttt{x}:十六进制
    \end{itemize}

    \textbf{4. 字符串格式化}
    \begin{itemize}
      \item \texttt{s}:字符串格式
      \item \verb|{:10s}|:字符串宽度10,左对齐
      \item \verb|{:>10s}|:右对齐,\verb|{:^10s}|:居中对齐
      \item \verb|{:*^10s}|:用*填充的居中对齐
    \end{itemize}

    \textbf{5. 格式化示例}
    \begin{itemize}
      \item \verb|print("{:.2f}".format(3.14159))| → 3.14
      \item \verb|print("{:10.2f}".format(3.14))| → \quad\quad\quad 3.14
      \item \verb|print("{:0>5d}".format(42))| → 00042
      \item \verb|print(f"{name:^10s}")| → 变量name居中显示
    \end{itemize}

  \end{mdframed}

  \subsection{第8题}
  定义类时实现了\texttt{\_\_pow\_\_()}方法,该类对象即可支持运算符**。(\quad)

  \begin{mdframed}[linewidth=1pt, linecolor=black]
    \textbf{\color{red}【笔记】}

    正确答案:正确 (T)

    解释:
    \begin{itemize}
      \item \texttt{\_\_pow\_\_()}是Python的魔法方法(magic method)
      \item 实现该方法后,类对象可以使用**运算符进行幂运算
      \item 调用\texttt{obj ** n}等价于调用\texttt{obj.\_\_pow\_\_(n)}
    \end{itemize}

    其他常用魔法方法:
    \begin{itemize}
      \item \texttt{\_\_add\_\_()}:支持+运算符
      \item \texttt{\_\_sub\_\_()}:支持-运算符
      \item \texttt{\_\_mul\_\_()}:支持*运算符
      \item \texttt{\_\_str\_\_()}:支持字符串转换
    \end{itemize}

  \end{mdframed}

  \subsection{第9题}
  通过对象和类名都可以调用类方法和静态方法。(\quad)

  \begin{mdframed}[linewidth=1pt, linecolor=black]
    \textbf{\color{red}【笔记】}

    正确答案:正确 (T)

    解释:
    \begin{itemize}
      \item 类方法(\texttt{@classmethod}):可通过类名和对象调用
      \item 静态方法(\texttt{@staticmethod}):可通过类名和对象调用
      \item 这两种方法不依赖于特定的实例状态
    \end{itemize}

    方法调用方式对比:
    \begin{itemize}
      \item 实例方法:只能通过对象调用,需要\texttt{self}参数
      \item 类方法:类名.方法() 或 对象.方法(),接收\texttt{cls}参数
      \item 静态方法:类名.方法() 或 对象.方法(),无特殊参数要求
    \end{itemize}

    推荐做法:类方法和静态方法优先使用类名调用,语义更清晰。

    \textbf{类方法与静态方法详细区别:}

    \textbf{1. 定义方式}
    \begin{itemize}
      \item 类方法:使用\texttt{@classmethod}装饰器
      \item 静态方法:使用\texttt{@staticmethod}装饰器
    \end{itemize}

    \textbf{2. 参数区别}
    \begin{itemize}
      \item 类方法:第一个参数必须是\texttt{cls}(代表类本身)
      \item 静态方法:没有特殊的第一个参数要求
      \item 实例方法:第一个参数必须是\texttt{self}(代表实例本身)
    \end{itemize}

    \textbf{3. 访问权限}
    \begin{itemize}
      \item 类方法:可以访问类属性,不能直接访问实例属性
      \item 静态方法:不能访问类属性和实例属性
      \item 实例方法:可以访问类属性和实例属性
    \end{itemize}

    \textbf{4. 使用场景}
    \begin{itemize}
      \item 类方法:替代构造函数、操作类属性、工厂方法
      \item 静态方法:与类相关但不需要访问类/实例数据的工具函数
      \item 实例方法:操作具体实例的数据和行为
    \end{itemize}

    \textbf{5. 代码示例}
    \begin{lstlisting}[linewidth=\textwidth, breaklines=true, numbers=none]
class MyClass:
    class_var = "类属性"

    def __init__(self, value):
        self.instance_var = value

    @classmethod
    def class_method(cls):
        return f"访问: {cls.class_var}"

    @staticmethod
    def static_method():
        return "静态方法"

    def instance_method(self):
        return f"实例: {self.instance_var}"

# 调用方式
MyClass.class_method()    # 类方法
MyClass.static_method()   # 静态方法
obj = MyClass("值")
obj.class_method()        # 对象调用类方法
obj.static_method()       # 对象调用静态方法
    \end{lstlisting}
  \end{mdframed}

\section{填空题}

\subsection{第1题}
有如下Python程序:
\begin{lstlisting}
def f(a, b):
    if b == 0:
        print(a)
    else:
        f(b, a % b)
\end{lstlisting}
则\texttt{print(f(9,6))}的输出结果是\underline{\hspace{2cm}}、\underline{\hspace{2cm}}

\begin{mdframed}[linewidth=1pt, linecolor=black]

  \textbf{\color{red}【笔记】}

  正确答案:\underline{3}, \underline{None}

  解释:

  \qquad 这是一个经典的辗转相除法求最大公约数(GCD)的递归代码。
  当b为0时,a就等于gcd(a,b)的值。由于没有返回值,所以print(f(a,b))输出None。

\end{mdframed}

\subsection{第2题}
表达式\texttt{sorted(['abc', 'acd', 'ade'], key=lambda x:(x[0],x[2]))}的值为\\
\underline{\hspace{8cm}}

\begin{mdframed}[linewidth=1pt, linecolor=black]

  \textbf{\color{red}【笔记】}

  正确答案:\underline{['abc', 'acd', 'ade']}

  解释:

  \qquad sorted函数使用key参数来指定排序的依据。这里的key是lambda x:(x[0],x[2]),
  意味着对每个字符串取第0个字符和第2个字符组成元组作为排序键。

  \qquad 分析每个字符串的排序键:
  \begin{itemize}
    \item 'abc': (x[0], x[2]) = ('a', 'c')
    \item 'acd': (x[0], x[2]) = ('a', 'd')
    \item 'ade': (x[0], x[2]) = ('a', 'e')
  \end{itemize}

  \qquad 由于第0个字符都是'a',按第2个字符排序:'c' < 'd' < 'e',
  所以原顺序已经是正确的排序结果。

\end{mdframed}

\subsection{第3题}
表达式\verb|'The first:{1:.3f}, the second is {0:b}'.format(2,3.1415926)|的值为\\
\underline{\hspace{8cm}}

\begin{mdframed}[linewidth=1pt, linecolor=black]

  \textbf{\color{red}【笔记】}

  正确答案:\underline{'The first:3.142, the second is 10'}

  解释:

  \qquad 这是字符串的format方法格式化,format(2, 3.1415926)传入两个参数:
  \begin{itemize}
    \item 参数0:2
    \item 参数1:3.1415926
  \end{itemize}

  \qquad 格式化说明:
  \begin{itemize}
    \item \{1:.3f\}:取第1个参数3.1415926,格式化为保留3位小数的浮点数 → 3.142
    \item \{0:b\}:取第0个参数2,格式化为二进制 → 10
  \end{itemize}

  \qquad 因此最终结果为:'The first:3.142, the second is 10'

\end{mdframed}

\subsection{第4题}
假设正则表达式模块re已导入,那么表达式\texttt{re.sub('\textbackslash d+', '1', 'a12345bbbbb67c890d0e')}的值为\\
\underline{\hspace{8cm}}

\begin{mdframed}[linewidth=1pt, linecolor=black]

  \textbf{\color{red}【笔记】}

  正确答案:\underline{'a1bbbbb1c1d1e'}

  解释:

  \qquad re.sub(pattern, replacement, string)函数用于字符串替换:
  \begin{itemize}
    \item pattern: '\textbackslash d+' 即 \textbackslash d+,匹配一个或多个连续数字
    \item replacement: '1',替换成的字符串
    \item string: 'a12345bbbbb67c890d0e',要处理的原字符串
  \end{itemize}

  \qquad 匹配过程:
  \begin{itemize}
    \item 原字符串:'a12345bbbbb67c890d0e'
    \item \textbackslash d+ 匹配到的数字序列:12345, 67, 890, 0
    \item 每个数字序列都被替换为'1'
  \end{itemize}

  \qquad 替换结果:'a1bbbbb1c1d1e'

  \textbf{re模块的常见函数及用法}

  \begin{itemize}
    \item \textbf{re.match(pattern, string[, flags])}:从字符串开头开始匹配
      \begin{itemize}
        \item 只匹配字符串开头,如果开头不符合则返回None
        \item 示例:\texttt{re.match(r'\textbackslash d+', '123abc')} → 匹配到'123'
        \item 示例:\texttt{re.match(r'\textbackslash d+', 'abc123')} → 返回None
      \end{itemize}

    \item \textbf{re.search(pattern, string[, flags])}:在整个字符串中搜索第一个匹配
      \begin{itemize}
        \item 扫描整个字符串,返回第一个匹配的结果
        \item 示例:\texttt{re.search(r'\textbackslash d+', 'abc123def456')} → 匹配到'123'
      \end{itemize}

    \item \textbf{re.findall(pattern, string[, flags])}:查找所有匹配项
      \begin{itemize}
        \item 返回一个列表,包含所有匹配的字符串
        \item 示例:\texttt{re.findall(r'\textbackslash d+', 'a12b34c56')} → ['12', '34', '56']
      \end{itemize}

    \item \textbf{re.finditer(pattern, string[, flags])}:返回匹配对象的迭代器
      \begin{itemize}
        \item 返回一个迭代器,每个元素是一个Match对象
        \item 适用于大文本,节省内存
        \item 示例:\texttt{for m in re.finditer(r'\textbackslash d+', 'a12b34'): print(m.group())}
      \end{itemize}

    \item \textbf{re.sub(pattern, repl, string[, count=0, flags=0])}:替换匹配项
      \begin{itemize}
        \item 将匹配的部分替换为指定字符串
        \item count参数指定最多替换次数,0表示全部替换
        \item 示例:\texttt{re.sub(r'\textbackslash d+', 'X', 'a12b34')} → 'aXbX'
      \end{itemize}

    \item \textbf{re.split(pattern, string[, maxsplit=0, flags=0])}:按模式分割字符串
      \begin{itemize}
        \item 使用正则表达式作为分隔符分割字符串
        \item 示例:\texttt{re.split(r'[,;]', 'a,b;c,d')} → ['a', 'b', 'c', 'd']
      \end{itemize}

    \item \textbf{re.compile(pattern[, flags])}:编译正则表达式
      \begin{itemize}
        \item 将正则表达式编译成Pattern对象,提高重复使用的效率
        \item 示例:\texttt{p = re.compile(r'\textbackslash d+'); p.findall('a12b34')} → ['12', '34']
      \end{itemize}
  \end{itemize}

  \textbf{常用的flags参数:}
  \begin{itemize}
    \item \texttt{re.I} 或 \texttt{re.IGNORECASE}:忽略大小写
    \item \texttt{re.M} 或 \texttt{re.MULTILINE}:多行模式,\^{}和\$匹配每行的开头和结尾
    \item \texttt{re.S} 或 \texttt{re.DOTALL}:使.匹配包括换行符在内的所有字符
    \item \texttt{re.X} 或 \texttt{re.VERBOSE}:忽略空白和注释,可以写更清晰的正则表达式
  \end{itemize}

\end{mdframed}

\subsection{第5题}
下面代码的功能是,随机生成20个介于[1,50]之间的整数,然后统计每个整数出现频率。请把缺少的代码补全。
\begin{lstlisting}
import random
x = [random.______(1, 50) for i in range(20)]
d = dict()
for i in x:
    d[i] = d.get(i, ______) + 1
for k, v in d.items():
    print(k, v)
\end{lstlisting}
上述横线处分别填写\underline{\hspace{3cm}}、\underline{\hspace{3cm}}

\begin{mdframed}[linewidth=1pt, linecolor=black]

  \textbf{\color{red}【笔记】}

  正确答案:\underline{randint}、\underline{0}

  解释:

  \qquad 第一个空:random.\underline{randint}(1, 50)
  \begin{itemize}
    \item randint(a, b)函数生成[a, b]范围内的随机整数(包含端点)
    \item 这里需要生成[1, 50]之间的整数,所以用randint
  \end{itemize}

  \qquad 第二个空:d.get(i, \underline{0})
  \begin{itemize}
    \item dict.get(key, default)方法:如果key存在返回对应值,否则返回default
    \item 统计频率时,如果数字第一次出现,频率应该是0+1=1
    \item 如果数字已经存在,频率就是原来的值+1
  \end{itemize}

  \qquad 代码逻辑:生成20个随机数→遍历每个数→统计每个数的出现次数→输出结果

\end{mdframed}

\section{简答题}

\subsection{第1题}
在Python的运算符中,|属于何种类型运算符?在集合set中|有什么作用?并分析逻辑运算符"or"的短路求值特性。

\begin{mdframed}[linewidth=1pt, linecolor=black]

  \textbf{\color{red}【笔记】}

  \textbf{答案:}

  \textbf{1. |运算符的类型:}
  \qquad |属于\textbf{位运算符}(按位或运算符)。它对整数的二进制位进行按位或运算。

  \textbf{2. 在集合set中|的作用:}
  \qquad 在集合中,|表示\textbf{并集运算}。例如:
  \begin{lstlisting}
set1 = {1, 2, 3}
set2 = {3, 4, 5}
result = set1 | set2  # 结果:{1, 2, 3, 4, 5}
  \end{lstlisting}

  \textbf{3. or的短路求值特性:}
  \qquad \textbf{短路求值}是指当第一个操作数能够确定整个表达式的结果时,就不再计算第二个操作数。

  \qquad 对于or运算符:
  \begin{itemize}
    \item 如果第一个操作数为True,则不评估第二个操作数,直接返回第一个操作数的值
    \item 如果第一个操作数为False,则返回第二个操作数的值
  \end{itemize}

  \qquad 示例:
  \begin{lstlisting}
def func():
    print("函数被调用了")
    return True

result = True or func()  # func()不会被调用,不会打印
  \end{lstlisting}

\end{mdframed}

\subsection{第2题}
\texttt{zip()}函数可以将两个或多个可迭代类型(元组、列表等)组合为一个关系对(元组),并返回包含这些元素的zip对象,该函数非常适合生成键值对。

请根据以下示例数据:
\begin{itemize}
  \item \verb|[|(1, 'A'), (2, 'B'), (3, 'C'), (4, 'D')\verb|]|
  \item \verb|{1: 'A', 2: 'B', 3: 'C', 4: 'D'}|(1分)
  \item \verb|[|'A', 'B', 'C', 'D'\verb|]|
\end{itemize}

\subsection{第3题}
请列举pip常用的5个命令及其作用。


\section{代码阅读题}

\subsection{第1题}
阅读下面的代码,并分析假设文件"E:$\backslash$writeTest.txt"不存在的情况下两段代码可能发生的问题。

\textbf{代码1:}
\begin{lstlisting}
fp = open(r'E:\writeTest.txt')
fp.write("python")
fp.close()
\end{lstlisting}

\textbf{代码2:}
\begin{lstlisting}
fp = open(r'E:\writeTest.txt', 'a+')
fp.write("python")
fp.close()
\end{lstlisting}

\begin{mdframed}[linewidth=1pt, linecolor=black]

  \textbf{\color{red}【笔记】}

  \textbf{问题分析:}

  \textbf{代码1存在的问题:}
  \begin{itemize}
    \item \texttt{open(r'E:\\writeTest.txt')}没有指定打开模式,默认为'r'(只读模式)
    \item 如果文件不存在,会抛出\texttt{FileNotFoundError}异常
    \item 即使文件存在,只读模式下调用\texttt{write()}方法也会抛出\texttt{io.UnsupportedOperation}异常
    \item 这段代码无法正常执行
  \end{itemize}

  \textbf{代码2的情况:}
  \begin{itemize}
    \item 使用'a+'模式(追加+读写模式)
    \item 如果文件不存在,会自动创建该文件
    \item 可以正常写入内容"python"
    \item 这段代码可以正常执行
  \end{itemize}

  \textbf{文件打开模式详细说明:}

  \textbf{基本模式:}
  \begin{itemize}
    \item \textbf{'r'}:只读模式(默认模式)
      \begin{itemize}
        \item 文件不存在时抛出FileNotFoundError
        \item 只能读取,不能写入
        \item 文件指针位于文件开头
      \end{itemize}

    \item \textbf{'w'}:写入模式
      \begin{itemize}
        \item 文件不存在时自动创建
        \item 文件存在时清空内容(覆盖)
        \item 只能写入,不能读取
        \item 文件指针位于文件开头
      \end{itemize}

    \item \textbf{'a'}:追加模式
      \begin{itemize}
        \item 文件不存在时自动创建
        \item 文件存在时保留原内容
        \item 只能写入,不能读取
        \item 文件指针位于文件末尾
      \end{itemize}
  \end{itemize}

  \textbf{复合模式(可读可写):}
  \begin{itemize}
    \item \textbf{'r+'}:读写模式
      \begin{itemize}
        \item 文件必须存在,否则报错
        \item 可读可写,文件指针位于开头
        \item 写入会覆盖对应位置的内容
      \end{itemize}

    \item \textbf{'w+'}:写读模式
      \begin{itemize}
        \item 文件不存在时创建,存在时清空
        \item 可读可写,文件指针位于开头
      \end{itemize}

    \item \textbf{'a+'}:追加读写模式
      \begin{itemize}
        \item 文件不存在时创建,存在时保留内容
        \item 可读可写,写入时指针自动移到末尾
        \item 读取时可以移动指针到任意位置
      \end{itemize}
  \end{itemize}

  \textbf{二进制模式(在上述模式后加'b'):}
  \begin{itemize}
    \item 'rb'、'wb'、'ab'、'r+b'、'w+b'、'a+b'
    \item 用于处理二进制文件(图片、视频、exe等)
    \item 返回bytes对象而不是字符串
  \end{itemize}

\end{mdframed}

\subsection{第2题}
代码分析

\textbf{(1)} 阅读下面代码,解释其功能并写出执行结果。
\begin{lstlisting}
def demo(*p):
    return sum(p)
print(demo(1, 2, 3, 4, 5))
print(demo(1, 2, 3))
\end{lstlisting}

\textbf{(2)} 阅读下面代码,解释其功能并写出执行结果。
\begin{lstlisting}
def Join(ls, sep=None):
    return (sep or ',').join(ls)
print(Join(['a', 'b', 'c']))
print(Join(['a', 'b', 'c'], ':'))
\end{lstlisting}

\begin{mdframed}[linewidth=1pt, linecolor=black]

  \textbf{\color{red}【笔记】}

  \textbf{(1) 代码分析:}

  \textbf{功能:}定义一个可变参数函数,计算所有传入参数的和。

  \textbf{代码解释:}
  \begin{itemize}
    \item \texttt{*p}表示可变参数,接收任意数量的位置参数,组成元组
    \item \texttt{sum(p)}计算元组中所有数字的和
    \item 函数返回计算结果
  \end{itemize}

  \textbf{执行结果:}
  \begin{lstlisting}
15
6
  \end{lstlisting}

  \textbf{(2) 代码分析:}

  \textbf{功能:}定义一个字符串连接函数,使用指定分隔符或默认逗号连接列表元素。

  \textbf{代码解释:}
  \begin{itemize}
    \item \texttt{sep=None}设置默认参数为None
    \item \texttt{(sep or ',')}利用短路求值:如果sep为None或False,则使用','
    \item \texttt{join(ls)}用分隔符连接列表中的字符串元素
  \end{itemize}

  \textbf{执行结果:}
  \begin{lstlisting}
a,b,c
a:b:c
  \end{lstlisting}

  \textbf{知识点总结:}
  \begin{itemize}
    \item 可变参数(\texttt{*args})的使用
    \item 默认参数和短路求值的应用
    \item 字符串\texttt{join()}方法的使用
  \end{itemize}

\end{mdframed}

\subsection{第3题}
给出如下代码:
\begin{lstlisting}
from random import randint
result = list()
while True:
    result.append(randint(1, 10))
    if len(result) == 20:
        break
print(result)
\end{lstlisting}

以上代码中,result为何种类型变量?程序是否能够正常执行,若不能,请解释原因;若能,请分析其执行结果。

\begin{mdframed}[linewidth=1pt, linecolor=black]

  \textbf{\color{red}【笔记】}

  \textbf{代码分析:}

  \textbf{1. result的类型:}
  \begin{itemize}
    \item \texttt{result = list()}创建了一个空列表
    \item 因此result是\textbf{列表(list)类型}变量
  \end{itemize}

  \textbf{2. 程序执行情况:}
  \begin{itemize}
    \item \textbf{程序可以正常执行}
    \item 代码语法正确,逻辑清晰
  \end{itemize}

  \textbf{3. 程序执行过程:}
  \begin{itemize}
    \item 创建空列表result
    \item 进入无限循环\texttt{while True}
    \item 每次循环生成一个[1,10]范围的随机整数并添加到列表
    \item 当列表长度达到20时,执行\texttt{break}跳出循环
    \item 打印包含20个随机数的列表
  \end{itemize}

  \textbf{4. 执行结果:}
  \begin{itemize}
    \item 输出:包含20个随机整数的列表
    \item 每个整数都在[1,10]范围内(包含1和10)
    \item 示例输出:[3, 7, 1, 9, 2, 5, 8, 4, 10, 6, 1, 3, 9, 7, 2, 8, 5, 4, 10, 6]
    \item 注意:由于是随机数,每次运行结果都不同
  \end{itemize}

  \textbf{知识点总结:}
  \begin{itemize}
    \item 列表的创建方法:\texttt{list()}和\texttt{[]}
    \item \texttt{append()}方法向列表末尾添加元素
    \item \texttt{len()}函数获取列表长度
    \item 无限循环\texttt{while True}配合\texttt{break}的使用
  \end{itemize}

\end{mdframed}

\subsection{第4题}
根据如下代码回答问题:
\begin{lstlisting}
class MyArray:
    def __init__(self, *args):
        if not args:
            self.__value = []
        else:
            for arg in args:
                self.__value = list(args)

    def __sub__(self, n):
        b = MyArray()
        b.__value = [item - n for item in self.__value]
        return b

    def __str__(self):
        return str(self.__value)

# 测试主程序
if __name__ == '__main__':
    m = MyArray(10, 3, 2, 5)
\end{lstlisting}

请回答以下问题:

\textbf{(1)} \texttt{\_\_init\_\_}中参数\texttt{*args}具有什么含义?

\textbf{(2)} 请说明\texttt{\_\_sub\_\_}方法的功能。并在测试主程序中调用该方法,使其输出为[6, -1, -2, 1],写出调用代码。

\begin{mdframed}[linewidth=1pt, linecolor=black]

  \textbf{\color{red}【笔记】}

  \textbf{代码分析:}

  \textbf{(1) *args的含义:}
  \begin{itemize}
    \item \texttt{*args}是可变参数,表示接收任意数量的位置参数
    \item 在函数内部,args是一个元组,包含所有传入的参数
    \item 允许创建MyArray对象时传入0个或多个参数
    \item 例如:\texttt{MyArray()}、\texttt{MyArray(1)}、\texttt{MyArray(1,2,3)}都是合法的
  \end{itemize}

  \textbf{构造函数逻辑:}
  \begin{itemize}
    \item 如果没有传入参数(\texttt{not args}),创建空列表
    \item 如果传入参数,将args转换为列表赋值给\texttt{self.\_\_value}
    \item 注意:代码中的for循环是多余的,每次都会重新赋值
  \end{itemize}

  \textbf{(2) \_\_sub\_\_方法功能:}
  \begin{itemize}
    \item \texttt{\_\_sub\_\_}是Python的魔术方法,重载减法运算符'-'
    \item 功能:将数组中每个元素都减去指定的数n
    \item 返回一个新的MyArray对象,不修改原对象
    \item 实现:\texttt{[item - n for item in self.\_\_value]}列表推导式
  \end{itemize}

  \textbf{调用代码分析:}
  \begin{itemize}
    \item 当前:\texttt{m = MyArray(10, 3, 2, 5)},即m.\_\_value = [10, 3, 2, 5]
    \item 要得到[6, -1, -2, 1],需要每个元素减去4:
      \begin{itemize}
        \item 10 - 4 = 6
        \item 3 - 4 = -1
        \item 2 - 4 = -2
        \item 5 - 4 = 1
      \end{itemize}
  \end{itemize}

  \textbf{调用代码:}
  \begin{center}
  \begin{lstlisting}[linewidth=\linewidth]
result = m - 4
print(result)  # 输出:[6, -1, -2, 1]
  \end{lstlisting}
  \end{center}
  \textbf{知识点总结:}
  \begin{itemize}
    \item 可变参数\texttt{*args}的使用
    \item Python魔术方法\texttt{\_\_sub\_\_}重载运算符
    \item 列表推导式的应用
    \item 私有属性命名约定(双下划线开头)
  \end{itemize}

\end{mdframed}

\section{代码编写题}

\subsection{第1题}
编程实现,由用户输入一个不多于5位的正整数,要求求出它是几位数以及逆序打印出该数。

\begin{mdframed}
  \textbf{\color{red}【笔记】}

    \begin{lstlisting}
n = int(input())
s = str(n)  # 数字转换成字符串
print(len(s))  # 输出长度
print(s[::-1])  # 输出反转后的字符串

    \end{lstlisting}
\end{mdframed}

\subsection{第2题}
循环从用户处获得一组数据,直到用户直接输入回车退出,打印输出所有数据的和。本题不考虑输入异常情况。

\begin{mdframed}
  \textbf{\color{red}【笔记】}

    \begin{lstlisting}
n = input()
tot = 0

while n:  # 空字符串为False
    tot += eval(n)  # 转换为int类型并累加
    n = input()  # 再输入

print(tot)
    \end{lstlisting}
\end{mdframed}

\subsection{第3题}
假设os模块已导入,那么请写出C:$\backslash$Windows文件夹中(无需遍历该文件夹下的子文件夹)所有扩展名为.exe的文件

\begin{mdframed}
  \textbf{\color{red}【笔记】}

  \textbf{方法一:使用os.listdir()和字符串方法}
  \begin{lstlisting}
import os

path = r'C:\Windows'
files = os.listdir(path)

for file in files:
    if file.endswith('.exe'):
        print(file)
  \end{lstlisting}

  \textbf{方法二:使用列表推导式}
  \begin{lstlisting}
import os

path = r'C:\Windows'
exe_files = [f for f in os.listdir(path) if f.endswith('.exe')]
for file in exe_files:
    print(file)
  \end{lstlisting}

  \textbf{方法三:使用os.path.splitext()}
  \begin{lstlisting}
import os

path = r'C:\Windows'
for file in os.listdir(path):
    name, ext = os.path.splitext(file)
    if ext.lower() == '.exe':
        print(file)
  \end{lstlisting}

  \textbf{代码解释:}
  \begin{itemize}
    \item \texttt{r'C:\\Windows'}:使用原始字符串避免转义问题
    \item \texttt{os.listdir(path)}:列出指定目录下的所有文件和文件夹
    \item \texttt{file.endswith('.exe')}:判断文件名是否以.exe结尾
    \item \texttt{os.path.splitext()}:分离文件名和扩展名
    \item \texttt{ext.lower()}:转换为小写,增加匹配的容错性
  \end{itemize}

  \textbf{注意事项:}
  \begin{itemize}
    \item 此代码只列出当前目录的文件,不会递归遍历子目录
    \item 需要有访问C:\\Windows目录的权限
    \item 实际使用时可能需要异常处理
  \end{itemize}

  \textbf{os模块的常用函数介绍}

  \textbf{1. 目录操作函数:}
  \begin{itemize}
    \item \textbf{os.getcwd()}:获取当前工作目录
    \begin{lstlisting}
import os
current_dir = os.getcwd()
print(current_dir)  # 输出当前工作目录的绝对路径
    \end{lstlisting}

    \item \textbf{os.chdir(path)}:改变当前工作目录
    \begin{lstlisting}
os.chdir('/home/user/documents')  # Linux/Mac
os.chdir(r'C:\Users\Documents')   # Windows
    \end{lstlisting}

    \item \textbf{os.listdir(path)}:列出指定目录下的文件和子目录
    \begin{lstlisting}
files = os.listdir('.')  # 列出当前目录
files = os.listdir('/tmp')  # 列出/tmp目录
    \end{lstlisting}

    \item \textbf{os.mkdir(path)}:创建单级目录
    \begin{lstlisting}
os.mkdir('new_folder')  # 在当前目录创建new_folder
os.mkdir('/tmp/test')   # 在/tmp下创建test目录
    \end{lstlisting}

    \item \textbf{os.makedirs(path)}:递归创建多级目录
    \begin{lstlisting}
os.makedirs('dir1/dir2/dir3')  # 创建多级目录
os.makedirs('dir1/dir2', exist_ok=True)  # 如果存在则不报错
    \end{lstlisting}

    \item \textbf{os.rmdir(path)}:删除空目录
    \begin{lstlisting}
os.rmdir('empty_folder')  # 只能删除空目录
    \end{lstlisting}

    \item \textbf{os.removedirs(path)}:递归删除空目录
    \begin{lstlisting}
os.removedirs('dir1/dir2/dir3')  # 从内到外删除空目录
    \end{lstlisting}
  \end{itemize}

  \textbf{2. 文件操作函数:}
  \begin{itemize}
    \item \textbf{os.remove(path)}:删除文件
    \begin{lstlisting}
os.remove('test.txt')  # 删除test.txt文件
    \end{lstlisting}

    \item \textbf{os.rename(src, dst)}:重命名文件或目录
    \begin{lstlisting}
os.rename('old_name.txt', 'new_name.txt')  # 重命名文件
os.rename('old_dir', 'new_dir')  # 重命名目录
    \end{lstlisting}

    \item \textbf{os.stat(path)}:获取文件或目录的状态信息
    \begin{lstlisting}
stat_info = os.stat('file.txt')
print(stat_info.st_size)  # 文件大小(字节)
print(stat_info.st_mtime)  # 最后修改时间
    \end{lstlisting}
  \end{itemize}

  \textbf{3. os.path模块(路径操作):}
  \begin{itemize}
    \item \textbf{os.path.exists(path)}:判断路径是否存在
    \begin{lstlisting}
if os.path.exists('file.txt'):
    print('文件存在')
    \end{lstlisting}

    \item \textbf{os.path.isfile(path)}:判断是否为文件
    \begin{lstlisting}
if os.path.isfile('test.txt'):
    print('这是一个文件')
    \end{lstlisting}

    \item \textbf{os.path.isdir(path)}:判断是否为目录
    \begin{lstlisting}
if os.path.isdir('/home'):
    print('这是一个目录')
    \end{lstlisting}

    \item \textbf{os.path.join(path1, path2, ...)}:连接路径
    \begin{lstlisting}
# 跨平台的路径连接
path = os.path.join('home', 'user', 'file.txt')
# Windows: home\user\file.txt
# Linux/Mac: home/user/file.txt
    \end{lstlisting}

    \item \textbf{os.path.split(path)}:分割路径和文件名
    \begin{lstlisting}
dir_name, file_name = os.path.split('/home/user/file.txt')
# dir_name = '/home/user'
# file_name = 'file.txt'
    \end{lstlisting}

    \item \textbf{os.path.splitext(path)}:分离文件名和扩展名
    \begin{lstlisting}
name, ext = os.path.splitext('document.pdf')
# name = 'document'
# ext = '.pdf'
    \end{lstlisting}

    \item \textbf{os.path.basename(path)}:获取文件名
    \begin{lstlisting}
file_name = os.path.basename('/home/user/file.txt')
# file_name = 'file.txt'
    \end{lstlisting}

    \item \textbf{os.path.dirname(path)}:获取目录名
    \begin{lstlisting}
dir_name = os.path.dirname('/home/user/file.txt')
# dir_name = '/home/user'
    \end{lstlisting}

    \item \textbf{os.path.abspath(path)}:获取绝对路径
    \begin{lstlisting}
abs_path = os.path.abspath('file.txt')
# 返回file.txt的绝对路径
    \end{lstlisting>

    \item \textbf{os.path.getsize(path)}:获取文件大小
    \begin{lstlisting}
size = os.path.getsize('file.txt')  # 返回字节数
    \end{lstlisting}
  \end{itemize}

  \textbf{4. 环境变量操作:}
  \begin{itemize}
    \item \textbf{os.environ}:环境变量字典
    \begin{lstlisting}
# 获取环境变量
home = os.environ.get('HOME')  # Linux/Mac
home = os.environ.get('USERPROFILE')  # Windows

# 设置环境变量
os.environ['MY_VAR'] = 'value'
    \end{lstlisting}

    \item \textbf{os.getenv(key, default=None)}:获取环境变量
    \begin{lstlisting}
path = os.getenv('PATH', '')  # 获取PATH,不存在返回空字符串
    \end{lstlisting}
  \end{itemize}

  \textbf{5. 系统相关:}
  \begin{itemize}
    \item \textbf{os.name}:操作系统名称
    \begin{lstlisting}
print(os.name)  # 'posix'(Linux/Mac) 或 'nt'(Windows)
    \end{lstlisting}

    \item \textbf{os.system(command)}:执行系统命令
    \begin{lstlisting}
os.system('ls -la')  # Linux/Mac
os.system('dir')     # Windows
    \end{lstlisting}

    \item \textbf{os.sep}:路径分隔符
    \begin{lstlisting}
print(os.sep)  # '/' (Linux/Mac) 或 '\' (Windows)
    \end{lstlisting}
  \end{itemize}

  \textbf{6. 遍历目录树 - os.walk():}
  \begin{lstlisting}
import os

# 遍历目录树
for root, dirs, files in os.walk('/path/to/directory'):
    print(f'当前目录: {root}')
    print(f'子目录: {dirs}')
    print(f'文件: {files}')

    # 处理每个文件
    for file in files:
        file_path = os.path.join(root, file)
        print(f'文件路径: {file_path}')
  \end{lstlisting}

  \textbf{7. 实用示例:}

  \textbf{示例1:获取目录下所有特定类型的文件}
  \begin{lstlisting}
import os

def find_files(directory, extension):
    """查找目录下所有指定扩展名的文件"""
    result = []
    for root, dirs, files in os.walk(directory):
        for file in files:
            if file.endswith(extension):
                result.append(os.path.join(root, file))
    return result

# 查找所有.py文件
python_files = find_files('.', '.py')
  \end{lstlisting}

  \textbf{示例2:创建目录结构}
  \begin{lstlisting}
import os

def create_project_structure(project_name):
    """创建项目目录结构"""
    dirs = [
        f'{project_name}/src',
        f'{project_name}/tests',
        f'{project_name}/docs',
        f'{project_name}/data'
    ]

    for dir_path in dirs:
        os.makedirs(dir_path, exist_ok=True)
        print(f'创建目录: {dir_path}')
  \end{lstlisting}

  \textbf{示例3:批量重命名文件}
  \begin{lstlisting}
import os

def batch_rename(directory, old_ext, new_ext):
    """批量修改文件扩展名"""
    for filename in os.listdir(directory):
        if filename.endswith(old_ext):
            old_path = os.path.join(directory, filename)
            new_name = filename.replace(old_ext, new_ext)
            new_path = os.path.join(directory, new_name)
            os.rename(old_path, new_path)
            print(f'重命名: {filename} -> {new_name}')
  \end{lstlisting}

  \textbf{注意事项:}
  \begin{itemize}
    \item 使用os.path.join()构建路径,确保跨平台兼容性
    \item 在删除文件或目录前,先检查是否存在
    \item 处理文件操作时要考虑异常处理
    \item 使用原始字符串(r'')处理Windows路径,避免转义问题
    \item os.makedirs()的exist\_ok=True参数可以避免目录已存在的错误
  \end{itemize}
\end{mdframed}

\subsection{第4题}
在游戏应用中,经常会判断鼠标是否点击了某个人物或道具,本题将模拟此场景,编程实现判断某个点是否在某个矩形内,具体要求如下:

\textbf{(1)} 实现Point类,类中有私有属性\_\_x,\_\_y,代表鼠标的坐标,在类中实现方法get\_x,set\_x,get\_y,set\_y,使其可以分别访问相应属性。(4分)

\textbf{(2)} 实现Rectangle类,类中有私有属性x,y,代表矩形的左上角坐标,私有属性width和height,代表矩形的宽度和高度。(4分)

\textbf{(3)} 在Rectangle类中实现方法contain,判断某个点(Point类的对象)是否包含在此矩形中(4分)

\textbf{测试代码如下:}
\begin{lstlisting}
p1 = Point(20, 25)  # 点p1的坐标为x=20, y=25
# 新建矩形对象rect,其左上角坐标x=10、y=15,矩形的宽度50、高度60
rect = Rectangle(10, 15, 50, 60)
print(rect.contain(p1))

p1.set_x(80)  # 设置点p1的x坐标为80
p1.set_y(90)  # 设置点p1的y坐标为90
print(rect.contain(p1))
\end{lstlisting}

\textbf{输出结果:}
\begin{lstlisting}
True
False
\end{lstlisting}

\begin{mdframed}
  \textbf{\color{red}【笔记】}

  \textbf{完整代码实现:}
  \begin{lstlisting}
class Point:
    def __init__(self, x, y):
        self.__x = x
        self.__y = y

    def set_x(self, x):
        self.__x = x

    def set_y(self, y):
        self.__y = y

    def get_x(self):
        return self.__x

    def get_y(self):
        return self.__y

class Rectangle:
    def __init__(self, x, y, width, height):
        self.__x = x
        self.__y = y
        self.__width = width
        self.__height = height

    def contain(self, point):
        if self.__x <= point.get_x() <= self.__x + self.__width \
        and self.__y <= point.get_y() <= self.__y + self.__height:
            return True
        return False
  \end{lstlisting}

  \textbf{Python类详解:}

  \textbf{1. 类的基本概念:}
  \begin{itemize}
    \item 类(Class)是对象的模板,定义了对象的属性和方法
    \item 对象(Object)是类的实例,具有类定义的属性和行为
    \item 使用关键字\texttt{class}定义类,类名通常采用驼峰命名法
  \end{itemize}

  \textbf{2. 构造函数\_\_init\_\_:}
  \begin{itemize}
    \item \texttt{\_\_init\_\_}是特殊方法,用于初始化对象
    \item 创建对象时自动调用,相当于其他语言的构造函数
    \item \texttt{self}参数代表对象本身,必须是第一个参数
    \item 可以接收参数来初始化对象的属性
  \end{itemize}

  \textbf{3. 属性访问控制:}
  \begin{lstlisting}
# 公有属性:可以直接访问
self.x = x          # 公有属性

# 私有属性:不能直接从外部访问
self.__x = x        # 私有属性(双下划线开头)

# 访问示例
p = Point(10, 20)
print(p.x)          # 如果x是公有属性,可以直接访问
# print(p.__x)      # 错误!私有属性不能直接访问
print(p.get_x())    # 通过getter方法访问私有属性
  \end{lstlisting}

  \textbf{4. Getter和Setter方法:}
  \begin{itemize}
    \item Getter方法:用于获取私有属性的值
    \item Setter方法:用于设置私有属性的值
    \item 提供了对私有属性的受控访问
    \item 可以在方法中添加验证逻辑
  \end{itemize}

  \textbf{5. 方法定义和调用:}
  \begin{lstlisting}
# 方法定义
def method_name(self, parameters):
    # 方法体
    return value

# 方法调用
obj = ClassName()
result = obj.method_name(arguments)
  \end{lstlisting}

  \textbf{6. 代码逻辑分析:}

  \textbf{Point类:}
  \begin{itemize}
    \item 私有属性\texttt{\_\_x}和\texttt{\_\_y}存储坐标
    \item 提供getter和setter方法访问坐标值
    \item 封装了坐标点的基本操作
  \end{itemize}

  \textbf{Rectangle类:}
  \begin{itemize}
    \item 私有属性\texttt{\_\_x}、\texttt{\_\_y}、\texttt{\_\_width}、\texttt{\_\_height}存储矩形的位置和尺寸信息
    \item \texttt{contain}方法判断点是否在矩形内部
    \item 判断条件:\texttt{\_\_x <= point\_x <= \_\_x + \_\_width}且\texttt{\_\_y <= point\_y <= \_\_y + \_\_height}
  \end{itemize}

  \textbf{7. 面向对象编程优势:}
  \begin{itemize}
    \item \textbf{封装}:将数据和操作封装在类中
    \item \textbf{抽象}:隐藏实现细节,只暴露必要接口
    \item \textbf{重用性}:类可以被多次实例化使用
    \item \textbf{维护性}:便于代码的维护和扩展
  \end{itemize}

  \textbf{8. 类装饰器详解:}

  \textbf{8.1 @property装饰器:}
  \begin{itemize}
    \item @property将方法转换为属性,提供更优雅的访问方式
    \item 可以像访问属性一样调用方法,无需加括号
    \item 提供了getter、setter、deleter的完整支持
    \item 比传统的get/set方法更加Pythonic
  \end{itemize}

  \textbf{8.2 使用@property重写Point类:}
  \begin{lstlisting}
class Point:
    def __init__(self, x, y):
        self.__x = x
        self.__y = y

    @property
    def x(self):
        """获取x坐标"""
        return self.__x

    @x.setter
    def x(self, value):
        """设置x坐标,可以添加验证"""
        if not isinstance(value, (int, float)):
            raise TypeError("坐标必须是数字")
        self.__x = value

    @x.deleter
    def x(self):
        """删除x坐标"""
        print("删除x坐标")
        del self.__x

    @property
    def y(self):
        """获取y坐标"""
        return self.__y

    @y.setter
    def y(self, value):
        """设置y坐标"""
        if not isinstance(value, (int, float)):
            raise TypeError("坐标必须是数字")
        self.__y = value

    @property
    def distance_from_origin(self):
        """计算到原点的距离(只读属性)"""
        return (self.__x ** 2 + self.__y ** 2) ** 0.5
  \end{lstlisting}

  \textbf{8.3 @property的使用示例:}
  \begin{lstlisting}
# 创建点对象
p = Point(3, 4)

# 使用@property装饰的方法,像属性一样访问
print(p.x)                    # 输出: 3 (调用getter)
print(p.y)                    # 输出: 4 (调用getter)
print(p.distance_from_origin) # 输出: 5.0 (只读属性)

# 使用setter设置值
p.x = 10                      # 调用setter
p.y = 20                      # 调用setter

# 验证功能
try:
    p.x = "invalid"           # 触发TypeError异常
except TypeError as e:
    print(e)                  # 输出: 坐标必须是数字

# 删除属性
del p.x                       # 调用deleter
  \end{lstlisting}

  \textbf{8.4 传统方法vs @property对比:}
  \begin{lstlisting}
# 传统getter/setter方法
point1 = Point(10, 20)
print(point1.get_x())         # 需要调用方法
point1.set_x(30)              # 需要调用方法

# 使用@property装饰器
point2 = Point(10, 20)
print(point2.x)               # 像访问属性一样
point2.x = 30                 # 像设置属性一样
  \end{lstlisting}

  \textbf{8.5 @property的优势:}
  \begin{itemize}
    \item \textbf{简洁性}:访问语法更加直观,无需get/set前缀
    \item \textbf{封装性}:可以在getter/setter中添加验证逻辑
    \item \textbf{兼容性}:可以将普通属性升级为property而不破坏接口
    \item \textbf{只读属性}:可以创建计算属性(如distance\_from\_origin)
    \item \textbf{延迟计算}:属性值可以在访问时动态计算
  \end{itemize}

  \textbf{8.6 其他常用装饰器:}
  \begin{lstlisting}
class MyClass:
    @staticmethod
    def static_method():
        """静态方法,不需要self参数,不依赖实例"""
        return "这是静态方法"

    @classmethod
    def class_method(cls):
        """类方法,接收cls参数,可以访问类属性"""
        return f"这是{cls.__name__}的类方法"

    @property
    def read_only_prop(self):
        """只读属性"""
        return "只读值"

# 使用示例
obj = MyClass()
print(MyClass.static_method())    # 直接通过类调用
print(MyClass.class_method())     # 通过类调用类方法
print(obj.read_only_prop)         # 访问只读属性
  \end{lstlisting}

  \textbf{8.7 实际应用场景:}
  \begin{itemize}
    \item \textbf{数据验证}:在setter中验证输入数据的有效性
    \item \textbf{计算属性}:根据其他属性动态计算值
    \item \textbf{缓存机制}:在属性中实现计算结果的缓存
    \item \textbf{兼容性}:将直接属性访问升级为方法调用
    \item \textbf{日志记录}:在属性访问时记录日志
  \end{itemize}

  \textbf{9. 扩展知识:}
  \begin{lstlisting}
# 类属性和实例属性
class MyClass:
    class_var = "类属性"    # 类属性,所有实例共享

    def __init__(self, value):
        self.instance_var = value  # 实例属性,每个实例独有

# 特殊方法(魔术方法)
class Point:
    def __str__(self):
        return f"Point({self.__x}, {self.__y})"

    def __repr__(self):
        return f"Point({self.__x}, {self.__y})"

    def __eq__(self, other):
        return self.__x == other.__x and self.__y == other.__y
  \end{lstlisting}

\end{mdframed}


\end{document}