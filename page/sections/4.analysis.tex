\section{简答题}

\subsection{第1题}
在Python的运算符中,|属于何种类型运算符?在集合set中|有什么作用?并分析逻辑运算符"or"的短路求值特性。

\begin{mdframed}[linewidth=1pt, linecolor=black]

  \textbf{\color{red}【笔记】}

  \textbf{答案:}

  \textbf{1. |运算符的类型:}
  \qquad |属于\textbf{位运算符}(按位或运算符)。它对整数的二进制位进行按位或运算。

  \textbf{2. 在集合set中|的作用:}
  \qquad 在集合中,|表示\textbf{并集运算}。例如:
  \begin{lstlisting}
set1 = {1, 2, 3}
set2 = {3, 4, 5}
result = set1 | set2  # 结果:{1, 2, 3, 4, 5}
  \end{lstlisting}

  \textbf{3. or的短路求值特性:}
  \qquad \textbf{短路求值}是指当第一个操作数能够确定整个表达式的结果时,就不再计算第二个操作数。

  \qquad 对于or运算符:
  \begin{itemize}
    \item 如果第一个操作数为True,则不评估第二个操作数,直接返回第一个操作数的值
    \item 如果第一个操作数为False,则返回第二个操作数的值
  \end{itemize}

  \qquad 示例:
  \begin{lstlisting}
def func():
    print("函数被调用了")
    return True

result = True or func()  # func()不会被调用,不会打印
  \end{lstlisting}

\end{mdframed}

\subsection{第2题}
\texttt{zip()}函数可以将两个或多个可迭代类型(元组、列表等)组合为一个关系对(元组),并返回包含这些元素的zip对象,该函数非常适合生成键值对。

请根据以下示例数据:
\begin{itemize}
  \item \verb|[|(1, 'A'), (2, 'B'), (3, 'C'), (4, 'D')\verb|]|
  \item \verb|{1: 'A', 2: 'B', 3: 'C', 4: 'D'}|(1分)
  \item \verb|[|'A', 'B', 'C', 'D'\verb|]|
\end{itemize}

\subsection{第3题}
请列举pip常用的5个命令及其作用。
