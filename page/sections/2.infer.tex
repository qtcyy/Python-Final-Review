\section{判断题}

\subsection{第1题}
定义函数时,即便该函数不需要接收任何参数,也必须保留一对空的圆括号来表示这是一个函数。(\quad)

\begin{mdframed}[linewidth=1pt, linecolor=black]
  \textbf{\color{red}【笔记】}

  正确答案:正确 (T)

  解释:
  \begin{itemize}
    \item Python函数定义语法:\texttt{def function\_name():}
    \item 即使函数不需要参数,圆括号 \texttt{()} 也是必须的
    \item 圆括号是函数定义的语法要求,用于区分函数和变量
    \item 示例:\texttt{def hello(): print("Hello")}
  \end{itemize}

\end{mdframed}

\subsection{第2题}
已知\texttt{ls=[2, 4, 6]},那么执行语句\texttt{ls.append(8)}之后,ls的内存地址不变。(\quad)

\begin{mdframed}[linewidth=1pt, linecolor=black]
  \textbf{\color{red}【笔记】}

  正确答案:正确 (T)

  解释:
  \begin{itemize}
    \item \texttt{append()}方法是原地修改操作(in-place operation)
    \item 列表对象本身不会被重新创建,只是在原有内存空间中添加新元素
    \item 可以用\texttt{id()}函数验证:\texttt{id(ls)}在append前后相同
    \item 类似的原地修改方法还有:\texttt{extend()}, \texttt{insert()}, \texttt{remove()}等
  \end{itemize}

  对比:重新赋值操作如\texttt{ls = ls + [8]}会创建新的列表对象,内存地址会改变。

\end{mdframed}

\subsection{第3题}
continue语句在一旦在循环结构里被执行,将使得当前循环的整个循环提前结束。(\quad)

\begin{mdframed}[linewidth=1pt, linecolor=black]
  \textbf{\color{red}【笔记】}

  正确答案:错误 (F)

  解释:
  \begin{itemize}
    \item \texttt{continue}语句只是跳过当前循环迭代,继续下一次迭代
    \item \texttt{continue}不会终止整个循环,循环仍会继续执行
    \item \texttt{break}语句才会提前终止整个循环
  \end{itemize}

  示例对比:
  \begin{itemize}
    \item 使用\texttt{continue}:跳过满足条件的迭代,继续其他迭代
    \item 使用\texttt{break}:一旦满足条件就立即退出整个循环
  \end{itemize}

\end{mdframed}

\subsection{第4题}
\texttt{a,b=b,a}可以实现a和b值的交换。(\quad)

\begin{mdframed}[linewidth=1pt, linecolor=black]
  \textbf{\color{red}【笔记】}

  正确答案:正确 (T)

  解释:
  \begin{itemize}
    \item Python支持多重赋值(multiple assignment)
    \item 右侧\texttt{b,a}先被打包成元组\texttt{(b,a)}
    \item 然后元组被解包赋值给左侧的\texttt{a,b}
    \item 这种方式简洁高效,无需临时变量
  \end{itemize}

  传统交换方式对比:
  \begin{itemize}
    \item 传统方式:\texttt{temp=a; a=b; b=temp}(需要临时变量)
    \item Python方式:\texttt{a,b=b,a}(一行代码完成)
  \end{itemize}

\end{mdframed}

\subsection{第5题}
函数是封装了一些独立的功能,可以直接调用,Python内置了许多函数,同时可以自建函数来使用。(\quad)

\begin{mdframed}[linewidth=1pt, linecolor=black]
  \textbf{\color{red}【笔记】}

  正确答案:正确 (T)

  解释:
  \begin{itemize}
    \item 函数是代码重用和模块化的基础
    \item 函数封装特定功能,提高代码可读性和可维护性
    \item Python内置函数如:\texttt{print()}, \texttt{len()}, \texttt{max()}, \texttt{min()}等
    \item 用户可以使用\texttt{def}关键字自定义函数
  \end{itemize}

  函数的优点:
  \begin{itemize}
    \item 代码重用:避免重复编写相同代码
    \item 模块化:将复杂问题分解为小问题
    \item 易于测试和调试
  \end{itemize}

\end{mdframed}

\subsection{第6题}
使用内置函数\texttt{open()}且以'w'模式打开的文件,文件指针默认指向文件尾。(\quad)

\begin{mdframed}[linewidth=1pt, linecolor=black]
  \textbf{\color{red}【笔记】}

  正确答案:错误 (F)

  解释:
  \begin{itemize}
    \item 'w'模式(写入模式)文件指针指向文件开头
    \item 'w'模式会清空原文件内容,从头开始写入
    \item 'a'模式(追加模式)文件指针才指向文件尾
  \end{itemize}

  文件打开模式对比:
  \begin{itemize}
    \item 'r':只读模式,指针在文件开头
    \item 'w':写入模式,清空文件,指针在文件开头
    \item 'a':追加模式,指针在文件末尾
    \item 'r+':读写模式,指针在文件开头
  \end{itemize}

\end{mdframed}

\subsection{第7题}
利用\texttt{print()}格式化输出,\verb|{2:f}|能够控制浮点数的小数点后保留两位。(\quad)

\begin{mdframed}[linewidth=1pt, linecolor=black]
  \textbf{\color{red}【笔记】}

  正确答案:错误 (F)

  解释:
  \begin{itemize}
    \item \verb|{2:f}|不能控制小数点后保留两位
    \item 2表示参数索引(第3个参数,从0开始计数)
    \item f表示浮点数格式,但没有指定精度
    \item 要保留两位小数需要使用\verb|{2:.2f}|
  \end{itemize}

  正确的格式化语法:
  \begin{itemize}
    \item \verb|{2:.2f}|:第3个参数保留2位小数
    \item \verb|{:.2f}|:当前位置参数保留2位小数
    \item \verb|{2:f}|:第3个参数浮点数格式,但精度由系统默认决定
  \end{itemize}

  \label{string:format}
  \textbf{Python print()格式化输出详解:}

  \textbf{1. 基本格式化语法}
  \begin{itemize}
    \item 基本形式:\verb|print("格式字符串".format(参数))|
    \item f-string形式:\verb|print(f"格式字符串{变量}")| (Python 3.6+)
    \item \%格式化:\verb|print("格式字符串" % 参数)| (较老的方式)
    \end{itemize}

    \textbf{2. 格式说明符详解}
    \begin{itemize}
      \item \verb|{索引:格式说明符}|
      \item 索引:0, 1, 2... 或省略(按顺序)
      \item 格式说明符:[填充字符][对齐][宽度][.精度][类型]
    \end{itemize}

    \textbf{3. 数字格式化类型}
    \begin{itemize}
      \item \texttt{d}:十进制整数
      \item \texttt{f}:浮点数(默认6位小数)
      \item \texttt{.nf}:浮点数保留n位小数
      \item \texttt{e}:科学计数法
      \item \texttt{g}:自动选择f或e格式
      \item \texttt{o}:八进制,\texttt{x}:十六进制
    \end{itemize}

    \textbf{4. 字符串格式化}
    \begin{itemize}
      \item \texttt{s}:字符串格式
      \item \verb|{:10s}|:字符串宽度10,左对齐
      \item \verb|{:>10s}|:右对齐,\verb|{:^10s}|:居中对齐
      \item \verb|{:*^10s}|:用*填充的居中对齐
    \end{itemize}

    \textbf{5. 格式化示例}
    \begin{itemize}
      \item \verb|print("{:.2f}".format(3.14159))| → 3.14
      \item \verb|print("{:10.2f}".format(3.14))| → \quad\quad\quad 3.14
      \item \verb|print("{:0>5d}".format(42))| → 00042
      \item \verb|print(f"{name:^10s}")| → 变量name居中显示
    \end{itemize}

  \end{mdframed}

  \subsection{第8题}
  定义类时实现了\texttt{\_\_pow\_\_()}方法,该类对象即可支持运算符**。(\quad)

  \begin{mdframed}[linewidth=1pt, linecolor=black]
    \textbf{\color{red}【笔记】}

    正确答案:正确 (T)

    解释:
    \begin{itemize}
      \item \texttt{\_\_pow\_\_()}是Python的魔法方法(magic method)
      \item 实现该方法后,类对象可以使用**运算符进行幂运算
      \item 调用\texttt{obj ** n}等价于调用\texttt{obj.\_\_pow\_\_(n)}
    \end{itemize}

    其他常用魔法方法:
    \begin{itemize}
      \item \texttt{\_\_add\_\_()}:支持+运算符
      \item \texttt{\_\_sub\_\_()}:支持-运算符
      \item \texttt{\_\_mul\_\_()}:支持*运算符
      \item \texttt{\_\_str\_\_()}:支持字符串转换
    \end{itemize}

  \end{mdframed}

  \subsection{第9题}
  通过对象和类名都可以调用类方法和静态方法。(\quad)

  \begin{mdframed}[linewidth=1pt, linecolor=black]
    \textbf{\color{red}【笔记】}

    正确答案:正确 (T)

    解释:
    \begin{itemize}
      \item 类方法(\texttt{@classmethod}):可通过类名和对象调用
      \item 静态方法(\texttt{@staticmethod}):可通过类名和对象调用
      \item 这两种方法不依赖于特定的实例状态
    \end{itemize}

    方法调用方式对比:
    \begin{itemize}
      \item 实例方法:只能通过对象调用,需要\texttt{self}参数
      \item 类方法:类名.方法() 或 对象.方法(),接收\texttt{cls}参数
      \item 静态方法:类名.方法() 或 对象.方法(),无特殊参数要求
    \end{itemize}

    推荐做法:类方法和静态方法优先使用类名调用,语义更清晰。

    \textbf{类方法与静态方法详细区别:}

    \textbf{1. 定义方式}
    \begin{itemize}
      \item 类方法:使用\texttt{@classmethod}装饰器
      \item 静态方法:使用\texttt{@staticmethod}装饰器
    \end{itemize}

    \textbf{2. 参数区别}
    \begin{itemize}
      \item 类方法:第一个参数必须是\texttt{cls}(代表类本身)
      \item 静态方法:没有特殊的第一个参数要求
      \item 实例方法:第一个参数必须是\texttt{self}(代表实例本身)
    \end{itemize}

    \textbf{3. 访问权限}
    \begin{itemize}
      \item 类方法:可以访问类属性,不能直接访问实例属性
      \item 静态方法:不能访问类属性和实例属性
      \item 实例方法:可以访问类属性和实例属性
    \end{itemize}

    \textbf{4. 使用场景}
    \begin{itemize}
      \item 类方法:替代构造函数、操作类属性、工厂方法
      \item 静态方法:与类相关但不需要访问类/实例数据的工具函数
      \item 实例方法:操作具体实例的数据和行为
    \end{itemize}

    \textbf{5. 代码示例}
    \begin{lstlisting}[linewidth=\textwidth, breaklines=true, numbers=none]
class MyClass:
    class_var = "类属性"

    def __init__(self, value):
        self.instance_var = value

    @classmethod
    def class_method(cls):
        return f"访问: {cls.class_var}"

    @staticmethod
    def static_method():
        return "静态方法"

    def instance_method(self):
        return f"实例: {self.instance_var}"

# 调用方式
MyClass.class_method()    # 类方法
MyClass.static_method()   # 静态方法
obj = MyClass("值")
obj.class_method()        # 对象调用类方法
obj.static_method()       # 对象调用静态方法
    \end{lstlisting}
  \end{mdframed}