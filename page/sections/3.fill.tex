\section{填空题}

\subsection{第1题}
有如下Python程序:
\begin{lstlisting}
def f(a, b):
    if b == 0:
        print(a)
    else:
        f(b, a % b)
\end{lstlisting}
则\texttt{print(f(9,6))}的输出结果是\underline{\hspace{2cm}}、\underline{\hspace{2cm}}

\begin{mdframed}[linewidth=1pt, linecolor=black]

  \textbf{\color{red}【笔记】}

  正确答案:\underline{3}, \underline{None}

  解释:

  \qquad 这是一个经典的辗转相除法求最大公约数(GCD)的递归代码。
  当b为0时,a就等于gcd(a,b)的值。由于没有返回值,所以print(f(a,b))输出None。

\end{mdframed}

\subsection{第2题}
表达式\texttt{sorted(['abc', 'acd', 'ade'], key=lambda x:(x[0],x[2]))}的值为\\
\underline{\hspace{8cm}}

\begin{mdframed}[linewidth=1pt, linecolor=black]

  \textbf{\color{red}【笔记】}

  正确答案:\underline{['abc', 'acd', 'ade']}

  解释:

  \qquad sorted函数使用key参数来指定排序的依据。这里的key是lambda x:(x[0],x[2]),
  意味着对每个字符串取第0个字符和第2个字符组成元组作为排序键。

  \qquad 分析每个字符串的排序键:
  \begin{itemize}
    \item 'abc': (x[0], x[2]) = ('a', 'c')
    \item 'acd': (x[0], x[2]) = ('a', 'd')
    \item 'ade': (x[0], x[2]) = ('a', 'e')
  \end{itemize}

  \qquad 由于第0个字符都是'a',按第2个字符排序:'c' < 'd' < 'e',
  所以原顺序已经是正确的排序结果。

\end{mdframed}

\subsection{第3题}
表达式\verb|'The first:{1:.3f}, the second is {0:b}'.format(2,3.1415926)|的值为\\
\underline{\hspace{8cm}}

\begin{mdframed}[linewidth=1pt, linecolor=black]

  \textbf{\color{red}【笔记】}

  正确答案:\underline{'The first:3.142, the second is 10'}

  解释:

  \qquad 这是字符串的format方法格式化,format(2, 3.1415926)传入两个参数:
  \begin{itemize}
    \item 参数0:2
    \item 参数1:3.1415926
  \end{itemize}

  \qquad 格式化说明:
  \begin{itemize}
    \item \{1:.3f\}:取第1个参数3.1415926,格式化为保留3位小数的浮点数 → 3.142
    \item \{0:b\}:取第0个参数2,格式化为二进制 → 10
  \end{itemize}

  \qquad 因此最终结果为:'The first:3.142, the second is 10'

  详细字符串格式化解释在\ref{string:format}
\end{mdframed}

\subsection{第4题}
假设正则表达式模块re已导入,那么表达式\texttt{re.sub('\textbackslash d+', '1', 'a12345bbbbb67c890d0e')}的值为\\
\underline{\hspace{8cm}}

\begin{mdframed}[linewidth=1pt, linecolor=black]

  \textbf{\color{red}【笔记】}

  正确答案:\underline{'a1bbbbb1c1d1e'}

  解释:

  \qquad re.sub(pattern, replacement, string)函数用于字符串替换:
  \begin{itemize}
    \item pattern: '\textbackslash d+' 即 \textbackslash d+,匹配一个或多个连续数字
    \item replacement: '1',替换成的字符串
    \item string: 'a12345bbbbb67c890d0e',要处理的原字符串
  \end{itemize}

  \qquad 匹配过程:
  \begin{itemize}
    \item 原字符串:'a12345bbbbb67c890d0e'
    \item \textbackslash d+ 匹配到的数字序列:12345, 67, 890, 0
    \item 每个数字序列都被替换为'1'
  \end{itemize}

  \qquad 替换结果:'a1bbbbb1c1d1e'

  \textbf{re模块的常见函数及用法}

  \begin{itemize}
    \item \textbf{re.match(pattern, string[, flags])}:从字符串开头开始匹配
      \begin{itemize}
        \item 只匹配字符串开头,如果开头不符合则返回None
        \item 示例:\texttt{re.match(r'\textbackslash d+', '123abc')} → 匹配到'123'
        \item 示例:\texttt{re.match(r'\textbackslash d+', 'abc123')} → 返回None
      \end{itemize}

    \item \textbf{re.search(pattern, string[, flags])}:在整个字符串中搜索第一个匹配
      \begin{itemize}
        \item 扫描整个字符串,返回第一个匹配的结果
        \item 示例:\texttt{re.search(r'\textbackslash d+', 'abc123def456')} → 匹配到'123'
      \end{itemize}

    \item \textbf{re.findall(pattern, string[, flags])}:查找所有匹配项
      \begin{itemize}
        \item 返回一个列表,包含所有匹配的字符串
        \item 示例:\texttt{re.findall(r'\textbackslash d+', 'a12b34c56')} → ['12', '34', '56']
      \end{itemize}

    \item \textbf{re.finditer(pattern, string[, flags])}:返回匹配对象的迭代器
      \begin{itemize}
        \item 返回一个迭代器,每个元素是一个Match对象
        \item 适用于大文本,节省内存
        \item 示例:\texttt{for m in re.finditer(r'\textbackslash d+', 'a12b34'): print(m.group())}
      \end{itemize}

    \item \textbf{re.sub(pattern, repl, string[, count=0, flags=0])}:替换匹配项
      \begin{itemize}
        \item 将匹配的部分替换为指定字符串
        \item count参数指定最多替换次数,0表示全部替换
        \item 示例:\texttt{re.sub(r'\textbackslash d+', 'X', 'a12b34')} → 'aXbX'
      \end{itemize}

    \item \textbf{re.split(pattern, string[, maxsplit=0, flags=0])}:按模式分割字符串
      \begin{itemize}
        \item 使用正则表达式作为分隔符分割字符串
        \item 示例:\texttt{re.split(r'[,;]', 'a,b;c,d')} → ['a', 'b', 'c', 'd']
      \end{itemize}

    \item \textbf{re.compile(pattern[, flags])}:编译正则表达式
      \begin{itemize}
        \item 将正则表达式编译成Pattern对象,提高重复使用的效率
        \item 示例:\texttt{p = re.compile(r'\textbackslash d+'); p.findall('a12b34')} → ['12', '34']
      \end{itemize}
  \end{itemize}

  \textbf{常用的flags参数:}
  \begin{itemize}
    \item \texttt{re.I} 或 \texttt{re.IGNORECASE}:忽略大小写
    \item \texttt{re.M} 或 \texttt{re.MULTILINE}:多行模式,\^{}和\$匹配每行的开头和结尾
    \item \texttt{re.S} 或 \texttt{re.DOTALL}:使.匹配包括换行符在内的所有字符
    \item \texttt{re.X} 或 \texttt{re.VERBOSE}:忽略空白和注释,可以写更清晰的正则表达式
  \end{itemize}

\end{mdframed}

\subsection{第5题}
下面代码的功能是,随机生成20个介于[1,50]之间的整数,然后统计每个整数出现频率。请把缺少的代码补全。
\begin{lstlisting}
import random
x = [random.______(1, 50) for i in range(20)]
d = dict()
for i in x:
    d[i] = d.get(i, ______) + 1
for k, v in d.items():
    print(k, v)
\end{lstlisting}
上述横线处分别填写\underline{\hspace{3cm}}、\underline{\hspace{3cm}}

\begin{mdframed}[linewidth=1pt, linecolor=black]

  \textbf{\color{red}【笔记】}

  正确答案:\underline{randint}、\underline{0}

  解释:

  \qquad 第一个空:random.\underline{randint}(1, 50)
  \begin{itemize}
    \item randint(a, b)函数生成[a, b]范围内的随机整数(包含端点)
    \item 这里需要生成[1, 50]之间的整数,所以用randint
  \end{itemize}

  \qquad 第二个空:d.get(i, \underline{0})
  \begin{itemize}
    \item dict.get(key, default)方法:如果key存在返回对应值,否则返回default
    \item 统计频率时,如果数字第一次出现,频率应该是0+1=1
    \item 如果数字已经存在,频率就是原来的值+1
  \end{itemize}

  \qquad 代码逻辑:生成20个随机数→遍历每个数→统计每个数的出现次数→输出结果

\end{mdframed}