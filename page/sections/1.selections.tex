\section{单选题}

\subsection{第1题}
下面代码输出的结果是()
\begin{lstlisting}
def func(num):
    num += 1
a = 10
func(a)
print(a)
\end{lstlisting}

\begin{enumerate}[label=\Alph*.]
  \item 10
  \item 11
  \item int
  \item 程序执行错误
\end{enumerate}

\textbf{\color{red}【笔记】}

正确答案:A

Python中函数是传值调用,不影响原本的值。

\subsection{第2题}
切片操作\texttt{list(range(6))[::-1]}执行结果为()

\begin{enumerate}[label=\Alph*.]
  \item \verb|[|0, 2, 4, 6\verb|]|
  \item \verb|[|6, 5, 4, 3, 2, 1\verb|]|
  \item \verb|[|0, -1\verb|]|
  \item \verb|[|5, 4, 3, 2, 1, 0\verb|]|
\end{enumerate}

\begin{mdframed}[linewidth=1pt, linecolor=black]

  \textbf{\color{red}【笔记】}

  正确答案:D

  \texttt{list()}将对象转换成列表类型,\texttt{[::-1]}将列表反转。

\end{mdframed}

\subsection{第3题}
下列选项中可以获取Python整数类型帮助的是()

\begin{enumerate}[label=\Alph*.]
  \item help(float)
  \item dir(float)
  \item help(int)
  \item dir(str)
\end{enumerate}

\subsection{第4题}
以下选项中符合Python语言变量命名规则的是()

\begin{enumerate}[label=\Alph*.]
  \item it's
  \item 3C
  \item pass
  \item \_AI
\end{enumerate}

\begin{mdframed}[linewidth=1pt, linecolor=black]

  \textbf{\color{red}【笔记】}

  正确答案:D

  \begin{itemize}
    \item A选项中出现了\texttt{'}符号,在Python中的意思是字符串;
    \item B选项数字开头是不允许的;
    \item C选项是Python关键字;
    \item D选项符合Python变量名要求。
  \end{itemize}

\end{mdframed}

\subsection{第5题}
关于Python序列类型的通用操作符和函数,以下选项中描述错误的是()

\begin{enumerate}[label=\Alph*.]
  \item 如果x是s的元素,x in s 返回True
  \item 如果s是一个序列,s = [1,"kate",True],s[3] 返回True
  \item 如果s是一个序列,s = [1,"kate",True],s[-1] 返回True
  \item 如果x不是s的元素,x not in s 返回True
\end{enumerate}

\begin{mdframed}[linewidth=1pt, linecolor=black]

  \textbf{\color{red}【笔记】}

  正确答案:B

  列表s的长度为3,可以访问的索引为0,1,2,如果访问s[3]会报错。

\end{mdframed}

\subsection{第6题}
下列选项中不是Python保留字的是()

\begin{enumerate}[label=\Alph*.]
  \item False
  \item True
  \item do
  \item class
\end{enumerate}

\begin{mdframed}[linewidth=1pt, linecolor=black]

  \textbf{\color{red}【笔记】}

  正确答案:C

  do是C语言关键字,Python中无此关键字。

\end{mdframed}

\subsection{第7题}
Python 3.x语言中,以下表达式输出结果为66的选项是()

\begin{enumerate}[label=\Alph*.]
  \item print("6+6")
  \item print(6+6)
  \item print(eval("6+6"))
  \item print(eval("6" + "6"))
\end{enumerate}

\begin{mdframed}[linewidth=1pt, linecolor=black]

  \textbf{\color{red}【笔记】}

  正确答案:D

  \begin{itemize}
    \item A选项打印的是字符串"6+6";
    \item B选项打印的是6+6的值12;
    \item C选项使用eval将两个字符串'6'转换成数字6后相加,得到12并输出;
    \item D选项使用eval将字符串'66'转换成数字66并输出。
  \end{itemize}

\end{mdframed}

\subsection{第8题}
关于eval函数,以下选项中描述错误的是()

\begin{enumerate}[label=\Alph*.]
  \item eval函数的作用是将输入的字符串转为Python语句,并执行该语句
  \item 如果用户希望输入一个数字,并用程序对这个数字进行计算,可以采用eval(input())组合
  \item 执行eval("Hello")和执行eval("'Hello'")得到相同的结果
  \item 执行eval('123')输出123
\end{enumerate}

\begin{mdframed}[linewidth=1pt, linecolor=black]

  \textbf{\color{red}【笔记】}

  正确答案:C

  \textbf{eval函数详解:}

  \textbf{1. 基本功能:}
  \begin{itemize}
    \item eval()函数用于执行字符串表达式,并返回表达式的结果
    \item 语法:eval(expression, globals=None, locals=None)
    \item expression:要执行的字符串表达式
  \end{itemize}

  \textbf{2. 使用示例:}
    \begin{lstlisting}
# 数学表达式
eval("2 + 3")          # 返回 5
eval("2 * 3 + 1")      # 返回 7
eval("abs(-5)")        # 返回 5

# 字符串表达式
eval("'Hello'")        # 返回 'Hello'
eval("'3' + '4'")      # 返回 '34'

# 变量表达式
x = 10
eval("x + 5")          # 返回 15

# 列表和字典
eval("[1, 2, 3]")      # 返回 [1, 2, 3]
eval("{'a': 1, 'b': 2}")  # 返回 {'a': 1, 'b': 2}
    \end{lstlisting}

  \textbf{3. 错误示例分析:}
  \begin{itemize}
    \item eval("Hello"):会报错,因为Hello被当作变量名,但未定义
    \item eval("'Hello'"):正确,返回字符串'Hello'
    \item eval("123"):返回整数123
    \item eval("'123'"):返回字符串'123'
  \end{itemize}

  \textbf{4. 常见用法:}
    \begin{lstlisting}
# 用户输入处理
user_input = input("请输入表达式: ")
result = eval(user_input)  # 注意:有安全风险

# 配置文件解析
config_str = "{'debug': True, 'port': 8080}"
config = eval(config_str)
    \end{lstlisting}
\end{mdframed}

\subsection{第9题}
下面代码的输出结果是()
\begin{lstlisting}
d = {"大海":"蓝色", "天空":"灰色", "大地":"黑色"}
print(d["大地"], d.get("大地","黄色"), d.setdefault('草地','绿色'))
\end{lstlisting}

\begin{enumerate}[label=\Alph*.]
  \item 黑色 黑色 None
  \item 黑色 黄色 绿色
  \item 黑色 黄色 None
  \item 黑色 黑色 绿色
\end{enumerate}

\begin{mdframed}[linewidth=1pt, linecolor=black]

  \textbf{\color{red}【笔记】}

  正确答案:D

  \begin{itemize}
    \item 第一个d["大地"]在字典d中存在键,对应为"蓝色";
    \item 第二个d.get("大地","黄色"),其含义为获取键“大地”,如果不存在返回值“黄色”
    \item 第三个d.setdefault('草地','绿色'),key不存在,将key设置为default值,并返回default这个值。
  \end{itemize}

\end{mdframed}

\subsection{第10题}
以下哪个是Python中用于科学计算与可视化的第三方库()

\begin{enumerate}[label=\Alph*.]
  \item jieba
  \item scipy
  \item request
  \item random
\end{enumerate}

\subsection{第11题}
以下选项中,不是Python对文件的打开模式的是()

\begin{enumerate}[label=\Alph*.]
  \item 'r'
  \item 'c'
  \item 'w'
  \item '+'
\end{enumerate}

\begin{mdframed}[linewidth=1pt, linecolor=black]

  \textbf{\color{red}【笔记】}

  正确答案:B

  Python文件打开模式说明:
  \begin{itemize}
    \item 'r':只读模式(默认),文件必须存在
    \item 'w':写入模式,会覆盖原文件内容,如果文件不存在则创建
    \item '+':可读写模式,必须与其他模式组合使用,如'r+'、'w+'
    \item 'c':不是Python的文件打开模式,此选项为错误选项
  \end{itemize}

  其他常见的文件打开模式:
  \begin{itemize}
    \item 'a':追加模式,在文件末尾写入,如果文件不存在则创建
    \item 'x':独占创建模式,文件必须不存在
    \item 'b':二进制模式,与其他模式组合使用,如'rb'、'wb'
    \item 't':文本模式(默认),与其他模式组合使用,如'rt'、'wt'
  \end{itemize}

\end{mdframed}

\subsection{第12题}
关于下面代码中的变量x,以下选项中描述正确的是()
\begin{lstlisting}
fo = open(fname, "r")
for x in fo:
    print(x)
fo.close()
\end{lstlisting}

\begin{enumerate}[label=\Alph*.]
  \item 变量x表示文件中的一组字符
  \item 变量x表示文件中的一行字符
  \item 变量x表示文件中的一个字符
  \item 变量x表示文件中的全体字符
\end{enumerate}

\begin{mdframed}[linewidth=1pt, linecolor=black]

  \textbf{\color{red}【笔记】}

  正确答案:B

  代码分析:
  \begin{itemize}
    \item open(fname, "r")以只读模式打开文件,返回文件对象fo
    \item 在Python中,当使用for循环遍历文件对象时,每次迭代获得的是文件的一行内容
    \item 变量x在每次循环中存储文件的一行字符(包括行尾的换行符)
    \item print(x)会输出文件的每一行内容
    \item fo.close()关闭文件
  \end{itemize}

  知识点:
  \begin{itemize}
    \item 文件对象是可迭代的,for循环遍历文件对象时逐行读取
    \item 每行内容包含该行的所有字符以及行尾的换行符
    \item 这种方式适合处理大文件,因为每次只读取一行到内存中
  \end{itemize}

\end{mdframed}

\subsection{第13题}
以上代码输出结果为()
\begin{lstlisting}
for i in "Python":
    print(i, end=", ")
\end{lstlisting}

\begin{enumerate}[label=\Alph*.]
  \item P*y*t*h*o*n*
  \item P,y,t,h,o,n,
  \item P y t h o n
  \item Python
\end{enumerate}

\begin{mdframed}[linewidth=1pt, linecolor=black]

  \textbf{\color{red}【笔记】}

  正确答案:B

  代码分析:
  \begin{itemize}
    \item for循环遍历字符串"Python",每次取出一个字符赋值给变量i
    \item print(i, end=", ")中的end参数指定输出结束时不换行,而是输出", "
    \item 因此每个字符后面都会跟着逗号和空格
    \item 最终输出结果为:P, y, t, h, o, n,
  \end{itemize}

  知识点:print函数的end参数用于指定输出结束时的字符,默认为换行符\texttt{\\n}。

\end{mdframed}

\subsection{第14题}
执行Python语句\texttt{nums=set([1,2,2,3,3,3,4])}和\texttt{print(len(nums))}的结果是()

\begin{enumerate}[label=\Alph*.]
  \item 1
  \item 2
  \item 4
  \item 7
\end{enumerate}

\begin{mdframed}[linewidth=1pt, linecolor=black]

  \textbf{\color{red}【笔记】}

  正确答案:C

  代码分析:
  \begin{itemize}
    \item 列表[1,2,2,3,3,3,4]包含7个元素,其中有重复元素
    \item set()函数将列表转换为集合,集合具有元素唯一性,会自动去除重复元素
    \item 转换后的集合nums包含元素\{1,2,3,4\}
    \item len(nums)返回集合中元素的个数,即4
  \end{itemize}

  知识点:
  \begin{itemize}
    \item set是Python的内置数据类型,表示无序且不重复的元素集合
    \item set会自动去除重复元素,保持元素的唯一性
    \item len()函数返回序列或集合中元素的个数
  \end{itemize}

\end{mdframed}

\subsection{第15题}
如下代码的输出为()
\begin{lstlisting}
import re
s = 'a bc abc abbb abbbbbca'
re.findall('ab*', s)
\end{lstlisting}

\begin{enumerate}[label=\Alph*.]
  \item \verb|[|'ab', 'ab', 'ab'\verb|]|
  \item \verb|[|'ab', 'abbb', 'abbbb'\verb|]|
  \item \verb|[|'a', 'ab', 'abbb', 'abbbb', 'a'\verb|]|
  \item \verb|[|'a', 'ab', 'abbb', 'abbbb'\verb|]|
\end{enumerate}

\begin{mdframed}[linewidth=1pt, linecolor=black]

  \textbf{\color{red}【笔记】}

  正确答案:C

  \textbf{正则表达式详解:}

  \textbf{1. 基本概念:}
  \begin{itemize}
    \item 正则表达式(Regular Expression)是用于匹配字符串的强大工具
    \item 'ab*'表示匹配字母'a'后跟0个或多个字母'b'
    \item re.findall()返回字符串中所有匹配模式的子字符串列表
  \end{itemize}

  \textbf{2. 量词详解:}
  \begin{itemize}
    \item *:匹配前面字符0次或多次(贪婪匹配)
    \item +:匹配前面字符1次或多次
    \item ?:匹配前面字符0次或1次
    \item \{n\}:恰好匹配n次
    \item \{n,\}:至少匹配n次
    \item \{n,m\}:匹配n到m次
  \end{itemize}

  \textbf{3. 匹配过程分析:}
  \begin{lstlisting}
字符串:'a bc abc abbb abbbbbca'
模式:'ab*'

匹配过程:
位置0: 'a' -> 匹配'a'(0个b)✓
位置2: 'b' -> 不匹配(没有前导'a')
位置3: 'c' -> 不匹配
位置5: 'a' -> 开始匹配
位置6: 'b' -> 匹配'ab'(1个b)✓
位置7: 'c' -> 结束匹配
位置9: 'a' -> 开始匹配
位置10-12: 'bbb' -> 匹配'abbb'(3个b)✓
位置14: 'a' -> 开始匹配
位置15-18: 'bbbb' -> 匹配'abbbb'(4个b)✓
位置19: 'b' -> 继续匹配但遇到'c'
位置20: 'c' -> 结束匹配
位置21: 'a' -> 匹配'a'(0个b)✓

结果:['a', 'ab', 'abbb', 'abbbb', 'a']
  \end{lstlisting}

  \textbf{4. 正则表达式语法详解:}

  \textbf{4.1 基本字符类:}
  \begin{lstlisting}
.     # 匹配任意字符(除换行符)
\d    # 匹配数字[0-9]
\D    # 匹配非数字[^0-9]
\w    # 匹配字母、数字、下划线[a-zA-Z0-9_]
\W    # 匹配非字母数字下划线[^a-zA-Z0-9_]
\s    # 匹配空白字符(空格、制表符、换行符等)
\S    # 匹配非空白字符
  \end{lstlisting}

  \textbf{4.2 自定义字符类:}
  \begin{lstlisting}
[abc]     # 匹配a、b或c中的任意一个
[a-z]     # 匹配任意小写字母
[A-Z]     # 匹配任意大写字母
[0-9]     # 匹配任意数字(等同于\d)
[a-zA-Z]  # 匹配任意字母
[^abc]    # 匹配除a、b、c外的任意字符(取反)
[a-z0-9]  # 匹配小写字母或数字
  \end{lstlisting}

  \textbf{4.3 量词(重复次数):}
  \begin{lstlisting}
*         # 匹配前面的字符0次或多次
+         # 匹配前面的字符1次或多次
?         # 匹配前面的字符0次或1次
{n}       # 匹配前面的字符恰好n次
{n,}      # 匹配前面的字符至少n次
{n,m}     # 匹配前面的字符n到m次
*?        # 非贪婪匹配0次或多次
+?        # 非贪婪匹配1次或多次
??        # 非贪婪匹配0次或1次
  \end{lstlisting}

  \textbf{4.4 锚点(位置匹配):}
  \begin{lstlisting}
^         # 匹配字符串开头
$         # 匹配字符串结尾
\b        # 匹配单词边界
\B        # 匹配非单词边界
\A        # 匹配字符串绝对开头
\Z        # 匹配字符串绝对结尾
  \end{lstlisting}

  \textbf{4.5 分组和选择:}
  \begin{lstlisting}
()        # 分组,创建捕获组
(?:...)   # 非捕获组
|         # 或者(选择)
\1, \2    # 反向引用第1、2个捕获组
(?P<name>...)  # 命名捕获组
(?P=name)      # 引用命名捕获组
  \end{lstlisting}

  \textbf{4.6 特殊字符转义:}
  \begin{lstlisting}
\.        # 匹配字面意思的点号
\*        # 匹配字面意思的星号
\+        # 匹配字面意思的加号
\?        # 匹配字面意思的问号
\[        # 匹配字面意思的左方括号
\]        # 匹配字面意思的右方括号
\\        # 匹配字面意思的反斜杠
\^        # 匹配字面意思的插入符
\$        # 匹配字面意思的美元符号
  \end{lstlisting}

  \textbf{4.7 实用示例:}
  \begin{lstlisting}
# 匹配邮箱
r'\w+@\w+\.\w+'

# 匹配手机号(简单版)
r'1[3-9]\d{9}'

# 匹配IP地址
r'\d{1,3}\.\d{1,3}\.\d{1,3}\.\d{1,3}'

# 匹配HTML标签
r'<[^>]+>'

# 匹配中文字符
r'[\u4e00-\u9fa5]+'

# 匹配URL
r'https?://[^\s]+'
  \end{lstlisting}

  \textbf{5. re模块常用函数:}
  \begin{lstlisting}
import re

text = 'Python 3.8 is great!'

# 查找所有匹配
re.findall(r'\d+', text)        # ['3', '8']

# 搜索第一个匹配
match = re.search(r'\d+', text)  # 返回Match对象或None

# 从字符串开头匹配
match = re.match(r'Python', text)  # 匹配成功

# 替换
result = re.sub(r'\d+', 'X', text)  # 'Python X.X is great!'

# 分割
parts = re.split(r'\s+', text)   # ['Python', '3.8', 'is', 'great!']
  \end{lstlisting}

  \textbf{6. 贪婪匹配vs非贪婪匹配:}
  \begin{lstlisting}
text = 'abbbbbb'

# 贪婪匹配(默认)
re.findall('ab*', text)      # ['abbbbbb'] - 匹配尽可能多的b

# 非贪婪匹配
re.findall('ab*?', text)     # ['a'] - 匹配尽可能少的b
  \end{lstlisting}

  \textbf{知识点总结:}
  \begin{itemize}
    \item 理解量词的含义和用法
    \item 掌握正则表达式的匹配原理
    \item 熟悉re模块的常用函数
    \item 区分贪婪匹配和非贪婪匹配
  \end{itemize}

\end{mdframed}

\subsection{第16题}
以下不能创建一个元组的语句是()

\begin{enumerate}[label=\Alph*.]
  \item tup1 = ()
  \item tup2 = 1,
  \item tup3 = (1)
  \item dict4 = tuple("123")
\end{enumerate}

\begin{mdframed}[linewidth=1pt, linecolor=black]

  \textbf{\color{red}【笔记】}

  正确答案:C

  各选项分析:
  \begin{itemize}
    \item A选项:tup1 = () 创建一个空元组,正确
    \item B选项:tup2 = 1, 创建包含一个元素的元组,逗号是关键,正确
    \item C选项:tup3 = (1) 仅仅是给数字1加括号,结果是整数1,不是元组
    \item D选项:tuple("123") 将字符串转换为元组('1', '2', '3'),正确
  \end{itemize}

  知识点:
  \begin{itemize}
    \item 创建元组时,逗号是关键标识符,不是括号
    \item 单个元素的元组必须在元素后加逗号:(1,) 或 1,
    \item 空元组可以用 () 或 tuple() 创建
    \item tuple()函数可以将其他可迭代对象转换为元组
  \end{itemize}

\end{mdframed}

\subsection{第17题}
在PyCharm中运行整个程序的默认快捷键是()

\begin{enumerate}[label=\Alph*.]
  \item Shift+F10
  \item Ctrl+F10
  \item Shift+Ctrl+F10
  \item Shift+Ctrl+Enter
\end{enumerate}

\begin{mdframed}[linewidth=1pt, linecolor=black]

  \textbf{\color{red}【笔记】}

  正确答案:A

  PyCharm常用快捷键:
  \begin{itemize}
    \item Shift+F10:运行整个程序(Run)
    \item Ctrl+F10:运行当前文件
    \item Shift+Ctrl+F10:运行当前光标所在的配置
    \item Shift+Ctrl+Enter:完成当前语句
  \end{itemize}

  其他常用快捷键:
  \begin{itemize}
    \item Shift+F9:调试程序(Debug)
    \item Ctrl+Shift+F9:调试当前文件
    \item F8:单步执行(调试时)
    \item F9:继续执行(调试时)
  \end{itemize}

\end{mdframed}

\subsection{第18题}
下面的语句中()用来把路径path设置为默认路径

\begin{enumerate}[label=\Alph*.]
  \item os.chdir(path)
  \item os.mkdir(path)
  \item os.isdir(path)
  \item os.listdir(path)
\end{enumerate}

\begin{mdframed}[linewidth=1pt, linecolor=black]

  \textbf{\color{red}【笔记】}

  正确答案:A

  os模块常用函数说明:
  \begin{itemize}
    \item os.chdir(path):改变当前工作目录到指定路径,相当于cd命令
    \item os.mkdir(path):创建一个新目录
    \item os.isdir(path):判断路径是否为目录,返回True或False
    \item os.listdir(path):返回指定目录下的文件和目录列表
  \end{itemize}

  其他相关函数:
  \begin{itemize}
    \item os.getcwd():获取当前工作目录
    \item os.makedirs(path):递归创建目录
    \item os.path.exists(path):判断路径是否存在
    \item os.path.join():连接路径
  \end{itemize}

\end{mdframed}

\subsection{第19题}
在Python中,下列说法正确的是()

\begin{enumerate}[label=\Alph*.]
  \item 0xad是合法的十六进制数字表示形式
  \item 3+4j不是合法的Python表达式
  \item 可以使用if作为变量名
  \item 0o12f是合法的八进制数字
\end{enumerate}

\begin{mdframed}[linewidth=1pt, linecolor=black]

  \textbf{\color{red}【笔记】}

  正确答案:A

  各选项分析:
  \begin{itemize}
    \item A选项:0xad是合法的十六进制数字,0x前缀表示十六进制,ad是有效的hex数字
    \item B选项:3+4j是合法的复数表达式,j表示虚数单位
    \item C选项:if是Python保留字(关键字),不能用作变量名
    \item D选项:0o12f不合法,八进制数字只能包含0-7,f不是有效的八进制数字
  \end{itemize}

  Python数字表示形式:
  \begin{itemize}
    \item 十进制:直接写数字,如123
    \item 二进制:0b前缀,如0b1010
    \item 八进制:0o前缀,如0o777(只能包含0-7)
    \item 十六进制:0x前缀,如0xff(可包含0-9和a-f)
    \item 复数:实部+虚部j,如3+4j
  \end{itemize}

\end{mdframed}