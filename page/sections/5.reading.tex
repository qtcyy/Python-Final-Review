\section{代码阅读题}

\subsection{第1题}
阅读下面的代码,并分析假设文件"E:$\backslash$writeTest.txt"不存在的情况下两段代码可能发生的问题。

\textbf{代码1:}
\begin{lstlisting}
fp = open(r'E:\writeTest.txt')
fp.write("python")
fp.close()
\end{lstlisting}

\textbf{代码2:}
\begin{lstlisting}
fp = open(r'E:\writeTest.txt', 'a+')
fp.write("python")
fp.close()
\end{lstlisting}

\begin{mdframed}[linewidth=1pt, linecolor=black]

  \textbf{\color{red}【笔记】}

  \textbf{问题分析:}

  \textbf{代码1存在的问题:}
  \begin{itemize}
    \item \texttt{open(r'E:\\writeTest.txt')}没有指定打开模式,默认为'r'(只读模式)
    \item 如果文件不存在,会抛出\texttt{FileNotFoundError}异常
    \item 即使文件存在,只读模式下调用\texttt{write()}方法也会抛出\texttt{io.UnsupportedOperation}异常
    \item 这段代码无法正常执行
  \end{itemize}

  \textbf{代码2的情况:}
  \begin{itemize}
    \item 使用'a+'模式(追加+读写模式)
    \item 如果文件不存在,会自动创建该文件
    \item 可以正常写入内容"python"
    \item 这段代码可以正常执行
  \end{itemize}

  \textbf{文件打开模式详细说明:}

  \textbf{基本模式:}
  \begin{itemize}
    \item \textbf{'r'}:只读模式(默认模式)
      \begin{itemize}
        \item 文件不存在时抛出FileNotFoundError
        \item 只能读取,不能写入
        \item 文件指针位于文件开头
      \end{itemize}

    \item \textbf{'w'}:写入模式
      \begin{itemize}
        \item 文件不存在时自动创建
        \item 文件存在时清空内容(覆盖)
        \item 只能写入,不能读取
        \item 文件指针位于文件开头
      \end{itemize}

    \item \textbf{'a'}:追加模式
      \begin{itemize}
        \item 文件不存在时自动创建
        \item 文件存在时保留原内容
        \item 只能写入,不能读取
        \item 文件指针位于文件末尾
      \end{itemize}
  \end{itemize}

  \textbf{复合模式(可读可写):}
  \begin{itemize}
    \item \textbf{'r+'}:读写模式
      \begin{itemize}
        \item 文件必须存在,否则报错
        \item 可读可写,文件指针位于开头
        \item 写入会覆盖对应位置的内容
      \end{itemize}

    \item \textbf{'w+'}:写读模式
      \begin{itemize}
        \item 文件不存在时创建,存在时清空
        \item 可读可写,文件指针位于开头
      \end{itemize}

    \item \textbf{'a+'}:追加读写模式
      \begin{itemize}
        \item 文件不存在时创建,存在时保留内容
        \item 可读可写,写入时指针自动移到末尾
        \item 读取时可以移动指针到任意位置
      \end{itemize}
  \end{itemize}

  \textbf{二进制模式(在上述模式后加'b'):}
  \begin{itemize}
    \item 'rb'、'wb'、'ab'、'r+b'、'w+b'、'a+b'
    \item 用于处理二进制文件(图片、视频、exe等)
    \item 返回bytes对象而不是字符串
  \end{itemize}

\end{mdframed}

\subsection{第2题}
代码分析

\textbf{(1)} 阅读下面代码,解释其功能并写出执行结果。
\begin{lstlisting}
def demo(*p):
    return sum(p)
print(demo(1, 2, 3, 4, 5))
print(demo(1, 2, 3))
\end{lstlisting}

\textbf{(2)} 阅读下面代码,解释其功能并写出执行结果。
\begin{lstlisting}
def Join(ls, sep=None):
    return (sep or ',').join(ls)
print(Join(['a', 'b', 'c']))
print(Join(['a', 'b', 'c'], ':'))
\end{lstlisting}

\begin{mdframed}[linewidth=1pt, linecolor=black]

  \textbf{\color{red}【笔记】}

  \textbf{(1) 代码分析:}

  \textbf{功能:}定义一个可变参数函数,计算所有传入参数的和。

  \textbf{代码解释:}
  \begin{itemize}
    \item \texttt{*p}表示可变参数,接收任意数量的位置参数,组成元组
    \item \texttt{sum(p)}计算元组中所有数字的和
    \item 函数返回计算结果
  \end{itemize}

  \textbf{执行结果:}
  \begin{lstlisting}
15
6
  \end{lstlisting}

  \textbf{(2) 代码分析:}

  \textbf{功能:}定义一个字符串连接函数,使用指定分隔符或默认逗号连接列表元素。

  \textbf{代码解释:}
  \begin{itemize}
    \item \texttt{sep=None}设置默认参数为None
    \item \texttt{(sep or ',')}利用短路求值:如果sep为None或False,则使用','
    \item \texttt{join(ls)}用分隔符连接列表中的字符串元素
  \end{itemize}

  \textbf{执行结果:}
  \begin{lstlisting}
a,b,c
a:b:c
  \end{lstlisting}

  \textbf{知识点总结:}
  \begin{itemize}
    \item 可变参数(\texttt{*args})的使用
    \item 默认参数和短路求值的应用
    \item 字符串\texttt{join()}方法的使用
  \end{itemize}

\end{mdframed}

\subsection{第3题}
给出如下代码:
\begin{lstlisting}
from random import randint
result = list()
while True:
    result.append(randint(1, 10))
    if len(result) == 20:
        break
print(result)
\end{lstlisting}

以上代码中,result为何种类型变量?程序是否能够正常执行,若不能,请解释原因;若能,请分析其执行结果。

\begin{mdframed}[linewidth=1pt, linecolor=black]

  \textbf{\color{red}【笔记】}

  \textbf{代码分析:}

  \textbf{1. result的类型:}
  \begin{itemize}
    \item \texttt{result = list()}创建了一个空列表
    \item 因此result是\textbf{列表(list)类型}变量
  \end{itemize}

  \textbf{2. 程序执行情况:}
  \begin{itemize}
    \item \textbf{程序可以正常执行}
    \item 代码语法正确,逻辑清晰
  \end{itemize}

  \textbf{3. 程序执行过程:}
  \begin{itemize}
    \item 创建空列表result
    \item 进入无限循环\texttt{while True}
    \item 每次循环生成一个[1,10]范围的随机整数并添加到列表
    \item 当列表长度达到20时,执行\texttt{break}跳出循环
    \item 打印包含20个随机数的列表
  \end{itemize}

  \textbf{4. 执行结果:}
  \begin{itemize}
    \item 输出:包含20个随机整数的列表
    \item 每个整数都在[1,10]范围内(包含1和10)
    \item 示例输出:[3, 7, 1, 9, 2, 5, 8, 4, 10, 6, 1, 3, 9, 7, 2, 8, 5, 4, 10, 6]
    \item 注意:由于是随机数,每次运行结果都不同
  \end{itemize}

  \textbf{知识点总结:}
  \begin{itemize}
    \item 列表的创建方法:\texttt{list()}和\texttt{[]}
    \item \texttt{append()}方法向列表末尾添加元素
    \item \texttt{len()}函数获取列表长度
    \item 无限循环\texttt{while True}配合\texttt{break}的使用
  \end{itemize}

\end{mdframed}

\subsection{第4题}
根据如下代码回答问题:
\begin{lstlisting}
class MyArray:
    def __init__(self, *args):
        if not args:
            self.__value = []
        else:
            for arg in args:
                self.__value = list(args)

    def __sub__(self, n):
        b = MyArray()
        b.__value = [item - n for item in self.__value]
        return b

    def __str__(self):
        return str(self.__value)

# 测试主程序
if __name__ == '__main__':
    m = MyArray(10, 3, 2, 5)
\end{lstlisting}

请回答以下问题:

\textbf{(1)} \texttt{\_\_init\_\_}中参数\texttt{*args}具有什么含义?

\textbf{(2)} 请说明\texttt{\_\_sub\_\_}方法的功能。并在测试主程序中调用该方法,使其输出为[6, -1, -2, 1],写出调用代码。

\begin{mdframed}[linewidth=1pt, linecolor=black]

  \textbf{\color{red}【笔记】}

  \textbf{代码分析:}

  \textbf{(1) *args的含义:}
  \begin{itemize}
    \item \texttt{*args}是可变参数,表示接收任意数量的位置参数
    \item 在函数内部,args是一个元组,包含所有传入的参数
    \item 允许创建MyArray对象时传入0个或多个参数
    \item 例如:\texttt{MyArray()}、\texttt{MyArray(1)}、\texttt{MyArray(1,2,3)}都是合法的
  \end{itemize}

  \textbf{构造函数逻辑:}
  \begin{itemize}
    \item 如果没有传入参数(\texttt{not args}),创建空列表
    \item 如果传入参数,将args转换为列表赋值给\texttt{self.\_\_value}
    \item 注意:代码中的for循环是多余的,每次都会重新赋值
  \end{itemize}

  \textbf{(2) \_\_sub\_\_方法功能:}
  \begin{itemize}
    \item \texttt{\_\_sub\_\_}是Python的魔术方法,重载减法运算符'-'
    \item 功能:将数组中每个元素都减去指定的数n
    \item 返回一个新的MyArray对象,不修改原对象
    \item 实现:\texttt{[item - n for item in self.\_\_value]}列表推导式
  \end{itemize}

  \textbf{调用代码分析:}
  \begin{itemize}
    \item 当前:\texttt{m = MyArray(10, 3, 2, 5)},即m.\_\_value = [10, 3, 2, 5]
    \item 要得到[6, -1, -2, 1],需要每个元素减去4:
      \begin{itemize}
        \item 10 - 4 = 6
        \item 3 - 4 = -1
        \item 2 - 4 = -2
        \item 5 - 4 = 1
      \end{itemize}
  \end{itemize}

  \textbf{调用代码:}
  \begin{center}
  \begin{lstlisting}[linewidth=\linewidth]
result = m - 4
print(result)  # 输出:[6, -1, -2, 1]
  \end{lstlisting}
  \end{center}
  \textbf{知识点总结:}
  \begin{itemize}
    \item 可变参数\texttt{*args}的使用
    \item Python魔术方法\texttt{\_\_sub\_\_}重载运算符
    \item 列表推导式的应用
    \item 私有属性命名约定(双下划线开头)
  \end{itemize}

\end{mdframed}