\section{代码编写题}

\subsection{第1题}
编程实现,由用户输入一个不多于5位的正整数,要求求出它是几位数以及逆序打印出该数。

\begin{mdframed}
  \textbf{\color{red}【笔记】}

    \begin{lstlisting}
n = int(input())
s = str(n)  # 数字转换成字符串
print(len(s))  # 输出长度
print(s[::-1])  # 输出反转后的字符串

    \end{lstlisting}
\end{mdframed}

\subsection{第2题}
循环从用户处获得一组数据,直到用户直接输入回车退出,打印输出所有数据的和。本题不考虑输入异常情况。

\begin{mdframed}
  \textbf{\color{red}【笔记】}

    \begin{lstlisting}
n = input()
tot = 0

while n:  # 空字符串为False
    tot += eval(n)  # 转换为int类型并累加
    n = input()  # 再输入

print(tot)
    \end{lstlisting}
\end{mdframed}

\subsection{第3题}
假设os模块已导入,那么请写出C:$\backslash$Windows文件夹中(无需遍历该文件夹下的子文件夹)所有扩展名为.exe的文件

\begin{mdframed}
  \textbf{\color{red}【笔记】}

  \textbf{方法一:使用os.listdir()和字符串方法}
  \begin{lstlisting}
import os

path = r'C:\Windows'
files = os.listdir(path)

for file in files:
    if file.endswith('.exe'):
        print(file)
  \end{lstlisting}

  \textbf{方法二:使用列表推导式}
  \begin{lstlisting}
import os

path = r'C:\Windows'
exe_files = [f for f in os.listdir(path) if f.endswith('.exe')]
for file in exe_files:
    print(file)
  \end{lstlisting}

  \textbf{方法三:使用os.path.splitext()}
  \begin{lstlisting}
import os

path = r'C:\Windows'
for file in os.listdir(path):
    name, ext = os.path.splitext(file)
    if ext.lower() == '.exe':
        print(file)
  \end{lstlisting}

  \textbf{代码解释:}
  \begin{itemize}
    \item \texttt{r'C:\\Windows'}:使用原始字符串避免转义问题
    \item \texttt{os.listdir(path)}:列出指定目录下的所有文件和文件夹
    \item \texttt{file.endswith('.exe')}:判断文件名是否以.exe结尾
    \item \texttt{os.path.splitext()}:分离文件名和扩展名
    \item \texttt{ext.lower()}:转换为小写,增加匹配的容错性
  \end{itemize}

  \textbf{注意事项:}
  \begin{itemize}
    \item 此代码只列出当前目录的文件,不会递归遍历子目录
    \item 需要有访问C:\\Windows目录的权限
    \item 实际使用时可能需要异常处理
  \end{itemize}

  \textbf{os模块的常用函数介绍}

  \textbf{1. 目录操作函数:}
  \begin{itemize}
    \item \textbf{os.getcwd()}:获取当前工作目录
    \begin{lstlisting}
import os
current_dir = os.getcwd()
print(current_dir)  # 输出当前工作目录的绝对路径
    \end{lstlisting}

    \item \textbf{os.chdir(path)}:改变当前工作目录
    \begin{lstlisting}
os.chdir('/home/user/documents')  # Linux/Mac
os.chdir(r'C:\Users\Documents')   # Windows
    \end{lstlisting}

    \item \textbf{os.listdir(path)}:列出指定目录下的文件和子目录
    \begin{lstlisting}
files = os.listdir('.')  # 列出当前目录
files = os.listdir('/tmp')  # 列出/tmp目录
    \end{lstlisting}

    \item \textbf{os.mkdir(path)}:创建单级目录
    \begin{lstlisting}
os.mkdir('new_folder')  # 在当前目录创建new_folder
os.mkdir('/tmp/test')   # 在/tmp下创建test目录
    \end{lstlisting}

    \item \textbf{os.makedirs(path)}:递归创建多级目录
    \begin{lstlisting}
os.makedirs('dir1/dir2/dir3')  # 创建多级目录
os.makedirs('dir1/dir2', exist_ok=True)  # 如果存在则不报错
    \end{lstlisting}

    \item \textbf{os.rmdir(path)}:删除空目录
    \begin{lstlisting}
os.rmdir('empty_folder')  # 只能删除空目录
    \end{lstlisting}

    \item \textbf{os.removedirs(path)}:递归删除空目录
    \begin{lstlisting}
os.removedirs('dir1/dir2/dir3')  # 从内到外删除空目录
    \end{lstlisting}
  \end{itemize}

  \textbf{2. 文件操作函数:}
  \begin{itemize}
    \item \textbf{os.remove(path)}:删除文件
    \begin{lstlisting}
os.remove('test.txt')  # 删除test.txt文件
    \end{lstlisting}

    \item \textbf{os.rename(src, dst)}:重命名文件或目录
    \begin{lstlisting}
os.rename('old_name.txt', 'new_name.txt')  # 重命名文件
os.rename('old_dir', 'new_dir')  # 重命名目录
    \end{lstlisting}

    \item \textbf{os.stat(path)}:获取文件或目录的状态信息
    \begin{lstlisting}
stat_info = os.stat('file.txt')
print(stat_info.st_size)  # 文件大小(字节)
print(stat_info.st_mtime)  # 最后修改时间
    \end{lstlisting}
  \end{itemize}

  \textbf{3. os.path模块(路径操作):}
  \begin{itemize}
    \item \textbf{os.path.exists(path)}:判断路径是否存在
    \begin{lstlisting}
if os.path.exists('file.txt'):
    print('文件存在')
    \end{lstlisting}

    \item \textbf{os.path.isfile(path)}:判断是否为文件
    \begin{lstlisting}
if os.path.isfile('test.txt'):
    print('这是一个文件')
    \end{lstlisting}

    \item \textbf{os.path.isdir(path)}:判断是否为目录
    \begin{lstlisting}
if os.path.isdir('/home'):
    print('这是一个目录')
    \end{lstlisting}

    \item \textbf{os.path.join(path1, path2, ...)}:连接路径
    \begin{lstlisting}
# 跨平台的路径连接
path = os.path.join('home', 'user', 'file.txt')
# Windows: home\user\file.txt
# Linux/Mac: home/user/file.txt
    \end{lstlisting}

    \item \textbf{os.path.split(path)}:分割路径和文件名
    \begin{lstlisting}
dir_name, file_name = os.path.split('/home/user/file.txt')
# dir_name = '/home/user'
# file_name = 'file.txt'
    \end{lstlisting}

    \item \textbf{os.path.splitext(path)}:分离文件名和扩展名
    \begin{lstlisting}
name, ext = os.path.splitext('document.pdf')
# name = 'document'
# ext = '.pdf'
    \end{lstlisting}

    \item \textbf{os.path.basename(path)}:获取文件名
    \begin{lstlisting}
file_name = os.path.basename('/home/user/file.txt')
# file_name = 'file.txt'
    \end{lstlisting}

    \item \textbf{os.path.dirname(path)}:获取目录名
    \begin{lstlisting}
dir_name = os.path.dirname('/home/user/file.txt')
# dir_name = '/home/user'
    \end{lstlisting}

    \item \textbf{os.path.abspath(path)}:获取绝对路径
    \begin{lstlisting}
abs_path = os.path.abspath('file.txt')
# 返回file.txt的绝对路径
    \end{lstlisting>

    \item \textbf{os.path.getsize(path)}:获取文件大小
    \begin{lstlisting}
size = os.path.getsize('file.txt')  # 返回字节数
    \end{lstlisting}
  \end{itemize}

  \textbf{4. 环境变量操作:}
  \begin{itemize}
    \item \textbf{os.environ}:环境变量字典
    \begin{lstlisting}
# 获取环境变量
home = os.environ.get('HOME')  # Linux/Mac
home = os.environ.get('USERPROFILE')  # Windows

# 设置环境变量
os.environ['MY_VAR'] = 'value'
    \end{lstlisting}

    \item \textbf{os.getenv(key, default=None)}:获取环境变量
    \begin{lstlisting}
path = os.getenv('PATH', '')  # 获取PATH,不存在返回空字符串
    \end{lstlisting}
  \end{itemize}

  \textbf{5. 系统相关:}
  \begin{itemize}
    \item \textbf{os.name}:操作系统名称
    \begin{lstlisting}
print(os.name)  # 'posix'(Linux/Mac) 或 'nt'(Windows)
    \end{lstlisting}

    \item \textbf{os.system(command)}:执行系统命令
    \begin{lstlisting}
os.system('ls -la')  # Linux/Mac
os.system('dir')     # Windows
    \end{lstlisting}

    \item \textbf{os.sep}:路径分隔符
    \begin{lstlisting}
print(os.sep)  # '/' (Linux/Mac) 或 '\' (Windows)
    \end{lstlisting}
  \end{itemize}

  \textbf{6. 遍历目录树 - os.walk():}
  \begin{lstlisting}
import os

# 遍历目录树
for root, dirs, files in os.walk('/path/to/directory'):
    print(f'当前目录: {root}')
    print(f'子目录: {dirs}')
    print(f'文件: {files}')

    # 处理每个文件
    for file in files:
        file_path = os.path.join(root, file)
        print(f'文件路径: {file_path}')
  \end{lstlisting}

  \textbf{7. 实用示例:}

  \textbf{示例1:获取目录下所有特定类型的文件}
  \begin{lstlisting}
import os

def find_files(directory, extension):
    """查找目录下所有指定扩展名的文件"""
    result = []
    for root, dirs, files in os.walk(directory):
        for file in files:
            if file.endswith(extension):
                result.append(os.path.join(root, file))
    return result

# 查找所有.py文件
python_files = find_files('.', '.py')
  \end{lstlisting}

  \textbf{示例2:创建目录结构}
  \begin{lstlisting}
import os

def create_project_structure(project_name):
    """创建项目目录结构"""
    dirs = [
        f'{project_name}/src',
        f'{project_name}/tests',
        f'{project_name}/docs',
        f'{project_name}/data'
    ]

    for dir_path in dirs:
        os.makedirs(dir_path, exist_ok=True)
        print(f'创建目录: {dir_path}')
  \end{lstlisting}

  \textbf{示例3:批量重命名文件}
  \begin{lstlisting}
import os

def batch_rename(directory, old_ext, new_ext):
    """批量修改文件扩展名"""
    for filename in os.listdir(directory):
        if filename.endswith(old_ext):
            old_path = os.path.join(directory, filename)
            new_name = filename.replace(old_ext, new_ext)
            new_path = os.path.join(directory, new_name)
            os.rename(old_path, new_path)
            print(f'重命名: {filename} -> {new_name}')
  \end{lstlisting}

  \textbf{注意事项:}
  \begin{itemize}
    \item 使用os.path.join()构建路径,确保跨平台兼容性
    \item 在删除文件或目录前,先检查是否存在
    \item 处理文件操作时要考虑异常处理
    \item 使用原始字符串(r'')处理Windows路径,避免转义问题
    \item os.makedirs()的exist\_ok=True参数可以避免目录已存在的错误
  \end{itemize}
\end{mdframed}

\subsection{第4题}
在游戏应用中,经常会判断鼠标是否点击了某个人物或道具,本题将模拟此场景,编程实现判断某个点是否在某个矩形内,具体要求如下:

\textbf{(1)} 实现Point类,类中有私有属性\_\_x,\_\_y,代表鼠标的坐标,在类中实现方法get\_x,set\_x,get\_y,set\_y,使其可以分别访问相应属性。(4分)

\textbf{(2)} 实现Rectangle类,类中有私有属性x,y,代表矩形的左上角坐标,私有属性width和height,代表矩形的宽度和高度。(4分)

\textbf{(3)} 在Rectangle类中实现方法contain,判断某个点(Point类的对象)是否包含在此矩形中(4分)

\textbf{测试代码如下:}
\begin{lstlisting}
p1 = Point(20, 25)  # 点p1的坐标为x=20, y=25
# 新建矩形对象rect,其左上角坐标x=10、y=15,矩形的宽度50、高度60
rect = Rectangle(10, 15, 50, 60)
print(rect.contain(p1))

p1.set_x(80)  # 设置点p1的x坐标为80
p1.set_y(90)  # 设置点p1的y坐标为90
print(rect.contain(p1))
\end{lstlisting}

\textbf{输出结果:}
\begin{lstlisting}
True
False
\end{lstlisting}

\begin{mdframed}
  \textbf{\color{red}【笔记】}

  \textbf{完整代码实现:}
  \begin{lstlisting}
class Point:
    def __init__(self, x, y):
        self.__x = x
        self.__y = y

    def set_x(self, x):
        self.__x = x

    def set_y(self, y):
        self.__y = y

    def get_x(self):
        return self.__x

    def get_y(self):
        return self.__y

class Rectangle:
    def __init__(self, x, y, width, height):
        self.__x = x
        self.__y = y
        self.__width = width
        self.__height = height

    def contain(self, point):
        if self.__x <= point.get_x() <= self.__x + self.__width \
        and self.__y <= point.get_y() <= self.__y + self.__height:
            return True
        return False
  \end{lstlisting}

  \textbf{Python类详解:}

  \textbf{1. 类的基本概念:}
  \begin{itemize}
    \item 类(Class)是对象的模板,定义了对象的属性和方法
    \item 对象(Object)是类的实例,具有类定义的属性和行为
    \item 使用关键字\texttt{class}定义类,类名通常采用驼峰命名法
  \end{itemize}

  \textbf{2. 构造函数\_\_init\_\_:}
  \begin{itemize}
    \item \texttt{\_\_init\_\_}是特殊方法,用于初始化对象
    \item 创建对象时自动调用,相当于其他语言的构造函数
    \item \texttt{self}参数代表对象本身,必须是第一个参数
    \item 可以接收参数来初始化对象的属性
  \end{itemize}

  \textbf{3. 属性访问控制:}
  \begin{lstlisting}
# 公有属性:可以直接访问
self.x = x          # 公有属性

# 私有属性:不能直接从外部访问
self.__x = x        # 私有属性(双下划线开头)

# 访问示例
p = Point(10, 20)
print(p.x)          # 如果x是公有属性,可以直接访问
# print(p.__x)      # 错误!私有属性不能直接访问
print(p.get_x())    # 通过getter方法访问私有属性
  \end{lstlisting}

  \textbf{4. Getter和Setter方法:}
  \begin{itemize}
    \item Getter方法:用于获取私有属性的值
    \item Setter方法:用于设置私有属性的值
    \item 提供了对私有属性的受控访问
    \item 可以在方法中添加验证逻辑
  \end{itemize}

  \textbf{5. 方法定义和调用:}
  \begin{lstlisting}
# 方法定义
def method_name(self, parameters):
    # 方法体
    return value

# 方法调用
obj = ClassName()
result = obj.method_name(arguments)
  \end{lstlisting}

  \textbf{6. 代码逻辑分析:}

  \textbf{Point类:}
  \begin{itemize}
    \item 私有属性\texttt{\_\_x}和\texttt{\_\_y}存储坐标
    \item 提供getter和setter方法访问坐标值
    \item 封装了坐标点的基本操作
  \end{itemize}

  \textbf{Rectangle类:}
  \begin{itemize}
    \item 私有属性\texttt{\_\_x}、\texttt{\_\_y}、\texttt{\_\_width}、\texttt{\_\_height}存储矩形的位置和尺寸信息
    \item \texttt{contain}方法判断点是否在矩形内部
    \item 判断条件:\texttt{\_\_x <= point\_x <= \_\_x + \_\_width}且\texttt{\_\_y <= point\_y <= \_\_y + \_\_height}
  \end{itemize}

  \textbf{7. 面向对象编程优势:}
  \begin{itemize}
    \item \textbf{封装}:将数据和操作封装在类中
    \item \textbf{抽象}:隐藏实现细节,只暴露必要接口
    \item \textbf{重用性}:类可以被多次实例化使用
    \item \textbf{维护性}:便于代码的维护和扩展
  \end{itemize}

  \textbf{8. 类装饰器详解:}

  \textbf{8.1 @property装饰器:}
  \begin{itemize}
    \item @property将方法转换为属性,提供更优雅的访问方式
    \item 可以像访问属性一样调用方法,无需加括号
    \item 提供了getter、setter、deleter的完整支持
    \item 比传统的get/set方法更加Pythonic
  \end{itemize}

  \textbf{8.2 使用@property重写Point类:}
  \begin{lstlisting}
class Point:
    def __init__(self, x, y):
        self.__x = x
        self.__y = y

    @property
    def x(self):
        """获取x坐标"""
        return self.__x

    @x.setter
    def x(self, value):
        """设置x坐标,可以添加验证"""
        if not isinstance(value, (int, float)):
            raise TypeError("坐标必须是数字")
        self.__x = value

    @x.deleter
    def x(self):
        """删除x坐标"""
        print("删除x坐标")
        del self.__x

    @property
    def y(self):
        """获取y坐标"""
        return self.__y

    @y.setter
    def y(self, value):
        """设置y坐标"""
        if not isinstance(value, (int, float)):
            raise TypeError("坐标必须是数字")
        self.__y = value

    @property
    def distance_from_origin(self):
        """计算到原点的距离(只读属性)"""
        return (self.__x ** 2 + self.__y ** 2) ** 0.5
  \end{lstlisting}

  \textbf{8.3 @property的使用示例:}
  \begin{lstlisting}
# 创建点对象
p = Point(3, 4)

# 使用@property装饰的方法,像属性一样访问
print(p.x)                    # 输出: 3 (调用getter)
print(p.y)                    # 输出: 4 (调用getter)
print(p.distance_from_origin) # 输出: 5.0 (只读属性)

# 使用setter设置值
p.x = 10                      # 调用setter
p.y = 20                      # 调用setter

# 验证功能
try:
    p.x = "invalid"           # 触发TypeError异常
except TypeError as e:
    print(e)                  # 输出: 坐标必须是数字

# 删除属性
del p.x                       # 调用deleter
  \end{lstlisting}

  \textbf{8.4 传统方法vs @property对比:}
  \begin{lstlisting}
# 传统getter/setter方法
point1 = Point(10, 20)
print(point1.get_x())         # 需要调用方法
point1.set_x(30)              # 需要调用方法

# 使用@property装饰器
point2 = Point(10, 20)
print(point2.x)               # 像访问属性一样
point2.x = 30                 # 像设置属性一样
  \end{lstlisting}

  \textbf{8.5 @property的优势:}
  \begin{itemize}
    \item \textbf{简洁性}:访问语法更加直观,无需get/set前缀
    \item \textbf{封装性}:可以在getter/setter中添加验证逻辑
    \item \textbf{兼容性}:可以将普通属性升级为property而不破坏接口
    \item \textbf{只读属性}:可以创建计算属性(如distance\_from\_origin)
    \item \textbf{延迟计算}:属性值可以在访问时动态计算
  \end{itemize}

  \textbf{8.6 其他常用装饰器:}
  \begin{lstlisting}
class MyClass:
    @staticmethod
    def static_method():
        """静态方法,不需要self参数,不依赖实例"""
        return "这是静态方法"

    @classmethod
    def class_method(cls):
        """类方法,接收cls参数,可以访问类属性"""
        return f"这是{cls.__name__}的类方法"

    @property
    def read_only_prop(self):
        """只读属性"""
        return "只读值"

# 使用示例
obj = MyClass()
print(MyClass.static_method())    # 直接通过类调用
print(MyClass.class_method())     # 通过类调用类方法
print(obj.read_only_prop)         # 访问只读属性
  \end{lstlisting}

  \textbf{8.7 实际应用场景:}
  \begin{itemize}
    \item \textbf{数据验证}:在setter中验证输入数据的有效性
    \item \textbf{计算属性}:根据其他属性动态计算值
    \item \textbf{缓存机制}:在属性中实现计算结果的缓存
    \item \textbf{兼容性}:将直接属性访问升级为方法调用
    \item \textbf{日志记录}:在属性访问时记录日志
  \end{itemize}

  \textbf{9. 扩展知识:}
  \begin{lstlisting}
# 类属性和实例属性
class MyClass:
    class_var = "类属性"    # 类属性,所有实例共享

    def __init__(self, value):
        self.instance_var = value  # 实例属性,每个实例独有

# 特殊方法(魔术方法)
class Point:
    def __str__(self):
        return f"Point({self.__x}, {self.__y})"

    def __repr__(self):
        return f"Point({self.__x}, {self.__y})"

    def __eq__(self, other):
        return self.__x == other.__x and self.__y == other.__y
  \end{lstlisting}

\end{mdframed}
